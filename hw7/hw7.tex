\documentclass{sjtuarticle}
\allowdisplaybreaks[1]
\usepackage{array}
\usepackage{ntheorem}
\usepackage{float}
\usepackage{pgfplots}
\pgfplotsset{compat=newest}
\usepackage{bm}
\usepackage{booktabs}
\usepackage{subcaption}
\usepackage[colorlinks]{hyperref}

\def\dd{\mathrm{d}}

\title{作业7}
\author{李子龙\\123033910195}
\begin{document}
\maketitle

% 作业:P104: 1,2(1),4,5,7,8,10,11其中第1(4),2(1),8题修正及补充题见附件

\begin{itemize}
    \item[1.] \begin{solution}
    \begin{itemize}
        \item[(1)]
        \begin{equation*}
            \int_{-h}^h f(x)\dd x\approx A_{-1}f(-h)+A_0f(0)+A_1f(h)
        \end{equation*}
        根据 Simpson 公式,有
        \begin{align*}
            A_{-1} &= \frac{h-(-h)}{6} =\frac{h}{3} \\
            A_0 &= \frac{2(h-(-h))}{3} = \frac{4h}{3} \\
            A_1 &= \frac{h-(-h)}{6} = \frac{h}{3}
        \end{align*}
        Simpson 公式具有 3 阶代数精度。
        \item[(2)]
        \begin{equation*}
            \int_{-2h}^{2h}f(x)\dd x\approx A_{-1}f(-h)+A_0f(0)+A_1f(h)
        \end{equation*}
        设 $f(x)=1,x,x^2$,有
        \begin{equation*}
            \left\{
                \begin{aligned}
                4h &= A_{-1}+A_0+A_1 \\
                0 &= A_{-1}(-h)+A_1h \\
                \frac{16}{3}h^3 &= A_{-1}h^2+A_1h^2
                \end{aligned}
            \right.
        \end{equation*}
        解得
        \begin{equation*}
            A_{-1}=\frac{8}{3}h \quad A_0 = -\frac{4}{3}h \quad A_1 = \frac{8}{3}h
        \end{equation*}
        当 $f(x)=x^3$ 时,
        \begin{equation*}
            \int_{-2h}^{2h}x^3\dd x=\frac{1}{4}[(2h)^4-(-2h)^4]=0=A_{-1}(-h)^3+A_1h^3
        \end{equation*}
        当 $f(x)=x^4$ 时,
        \begin{equation*}
            \int_{-2h}^{2h}x^4\dd x=\frac{1}{5}[(2h)^5-(-2h)^5]=\frac{64}{5}h^5\neq A_{-1}(-h)^4+A_1h^4
        \end{equation*}
        所以它具有3阶代数精度。
        \item[(3)]
        \begin{equation*}
            \int_{-1}^1 f(x)\dd x\approx \frac{f(-1)+2f(x_1)+3f(x_2)}{3}
        \end{equation*}
        对 $f(x)=1,x,x^2$ 均能准确成立,有
        \begin{equation*}
            \left\{
                \begin{aligned}
                2 &= \frac{1}{3}(1+2+3) \\
                0 &= \frac{1}{3}(-1+2x_1+3x_2) \\
                \frac{2}{3} &= \frac{1}{3}(1+2x_1^2+3x_2^2)
                \end{aligned}
            \right.
        \end{equation*}
        解得
        \begin{equation*}
            \begin{cases}
                x_1=\frac{1-\sqrt{6}}{5}\\
                x_2=\frac{3+2\sqrt{6}}{15}
            \end{cases}\text{或}\quad
            \begin{cases}
                x_1=\frac{1+\sqrt{6}}{5}\\
                x_2=\frac{3-2\sqrt{6}}{15}
            \end{cases}
        \end{equation*}
        当 $f(x)=x^3$ 时,
        \begin{equation*}
            \frac{1}{3}(-1+2x_1^3+3x_2^3)\neq \int_{-1}^1 x^3\dd x=0
        \end{equation*}
        所以它具有2阶代数精度。
        \item[(4)]
        \begin{equation*}
            \int_{0}^h f(x)\dd x\approx \frac{h}{2}[f(0)+f(h)]+ah^2[f^\prime(0)-f^\prime(h)] 
        \end{equation*}
        对于 $f(x)=1, x, x^2$ 而言均能准确成立,有
        \begin{equation*}
            \begin{cases}
                h = h \\
                \frac{1}{2}h^2 = \frac{h^2}{2} \\
                \frac{1}{3}h^3 = \frac{h^3}{2}-2ah^3
            \end{cases}
        \end{equation*}
        解得
        \begin{equation*}
            a = \frac{1}{12}
        \end{equation*}
        当 $f(x)=x^3,x^4$ 时,
        \begin{align*}
            \frac{1}{4}h^4 &= \frac{1}{2}h^4 -3ah^4 \\
            \frac{1}{5}h^5 &\neq \frac{1}{2}h^5 -4ah^5
        \end{align*}
        故它具有3阶代数精度。
    \end{itemize}
    \end{solution}
    \item[2.]\begin{solution}
        \begin{itemize}
            \item[(1)] 使用复化梯形公式,
            \begin{equation*}
                \int_{a}^b f(x)\dd x \approx \frac{h}{2}[f(a)+2\sum_{k=1}^{n-1}f(x_k)+f(b)]
            \end{equation*}
            $n=8$,$x_k=\frac{k}{8}(k=0,1,\cdots,8)$,$h=\frac{1}{8}$
            \begin{equation*}
                f(x_k)=\frac{k/8}{4+(k/8)^2}=\frac{8k}{256+k^2}
            \end{equation*}
            \begin{align*}
                \int_{0}^1 \frac{x}{4+x^2}\dd x = \frac{1}{16}\left(\frac{1}{4}+2\sum_{k=1}^{7}\frac{8k}{256+k^2}+\frac{1}{5}\right)=0.12703
            \end{align*}
            使用复化Simpson公式,
            \begin{equation*}
                \int_a^b f(x)\dd x \approx \frac{h}{6}[f(a)+4\sum_{k=0}^{n-1}f(x_{k+\frac{1}{2}})+2\sum_{k=1}^{n-1}f(x_k)+f(b)]
            \end{equation*}
            $x_{k+\frac{1}{2}}=\frac{2k+1}{16}$,
            \begin{equation*}
                f(x_{k+\frac{1}{2}})=\frac{16\times (2k+1)}{256\times 4+(2k+1)^2}=\frac{32k+16}{4k^2+4k+1025}
            \end{equation*}
            有
            \begin{equation*}
                \int_{0}^1 \frac{x}{4+x^2}\dd x = \frac{1}{48}\left(\frac{1}{4}+4\sum_{k=0}^7\frac{32k+16}{4k^2+4k+1025}+2\sum_{k=1}^7\frac{8k}{256+k^2}+\frac{1}{5}\right)=0.11678
            \end{equation*}
        \end{itemize}
    \end{solution}
\end{itemize}

\end{document}