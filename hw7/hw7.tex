\documentclass{sjtuarticle}
\allowdisplaybreaks[1]
\usepackage{array}
\usepackage{ntheorem}
\usepackage{float}
\usepackage{pgfplots}
\pgfplotsset{compat=newest}
\usepackage{bm}
\usepackage{booktabs}
\usepackage{subcaption}
\usepackage[colorlinks]{hyperref}

\def\dd{\mathrm{d}}
\def\ee{\mathrm{e}}

\title{作业7}
\author{李子龙\\123033910195}
\begin{document}
\maketitle

% 作业:P104: 1,2(1),4,5,7,8,10,11其中第1(4),2(1),8题修正及补充题见附件

\begin{itemize}
    \item[1.] \begin{solution}
    \begin{itemize}
        \item[(1)]
        \begin{equation*}
            \int_{-h}^h f(x)\dd x\approx A_{-1}f(-h)+A_0f(0)+A_1f(h)
        \end{equation*}
        根据 Simpson 公式,有
        \begin{align*}
            A_{-1} &= \frac{h-(-h)}{6} =\frac{h}{3} \\
            A_0 &= \frac{2(h-(-h))}{3} = \frac{4h}{3} \\
            A_1 &= \frac{h-(-h)}{6} = \frac{h}{3}
        \end{align*}
        Simpson 公式具有 3 阶代数精度。
        \item[(2)]
        \begin{equation*}
            \int_{-2h}^{2h}f(x)\dd x\approx A_{-1}f(-h)+A_0f(0)+A_1f(h)
        \end{equation*}
        设 $f(x)=1,x,x^2$,有
        \begin{equation*}
            \left\{
                \begin{aligned}
                4h &= A_{-1}+A_0+A_1 \\
                0 &= A_{-1}(-h)+A_1h \\
                \frac{16}{3}h^3 &= A_{-1}h^2+A_1h^2
                \end{aligned}
            \right.
        \end{equation*}
        解得
        \begin{equation*}
            A_{-1}=\frac{8}{3}h \quad A_0 = -\frac{4}{3}h \quad A_1 = \frac{8}{3}h
        \end{equation*}
        当 $f(x)=x^3$ 时,
        \begin{equation*}
            \int_{-2h}^{2h}x^3\dd x=\frac{1}{4}[(2h)^4-(-2h)^4]=0=A_{-1}(-h)^3+A_1h^3
        \end{equation*}
        当 $f(x)=x^4$ 时,
        \begin{equation*}
            \int_{-2h}^{2h}x^4\dd x=\frac{1}{5}[(2h)^5-(-2h)^5]=\frac{64}{5}h^5\neq A_{-1}(-h)^4+A_1h^4
        \end{equation*}
        所以它具有3阶代数精度。
        \item[(3)]
        \begin{equation*}
            \int_{-1}^1 f(x)\dd x\approx \frac{f(-1)+2f(x_1)+3f(x_2)}{3}
        \end{equation*}
        对 $f(x)=1,x,x^2$ 均能准确成立,有
        \begin{equation*}
            \left\{
                \begin{aligned}
                2 &= \frac{1}{3}(1+2+3) \\
                0 &= \frac{1}{3}(-1+2x_1+3x_2) \\
                \frac{2}{3} &= \frac{1}{3}(1+2x_1^2+3x_2^2)
                \end{aligned}
            \right.
        \end{equation*}
        解得
        \begin{equation*}
            \begin{cases}
                x_1=\frac{1-\sqrt{6}}{5}\\
                x_2=\frac{3+2\sqrt{6}}{15}
            \end{cases}\text{或}\quad
            \begin{cases}
                x_1=\frac{1+\sqrt{6}}{5}\\
                x_2=\frac{3-2\sqrt{6}}{15}
            \end{cases}
        \end{equation*}
        当 $f(x)=x^3$ 时,
        \begin{equation*}
            \frac{1}{3}(-1+2x_1^3+3x_2^3)\neq \int_{-1}^1 x^3\dd x=0
        \end{equation*}
        所以它具有2阶代数精度。
        \item[(4)]
        \begin{equation*}
            \int_{0}^h f(x)\dd x\approx \frac{h}{2}[f(0)+f(h)]+ah^2[f^\prime(0)-f^\prime(h)] 
        \end{equation*}
        对于 $f(x)=1, x, x^2$ 而言均能准确成立,有
        \begin{equation*}
            \begin{cases}
                h = h \\
                \frac{1}{2}h^2 = \frac{h^2}{2} \\
                \frac{1}{3}h^3 = \frac{h^3}{2}-2ah^3
            \end{cases}
        \end{equation*}
        解得
        \begin{equation*}
            a = \frac{1}{12}
        \end{equation*}
        当 $f(x)=x^3,x^4$ 时,
        \begin{align*}
            \frac{1}{4}h^4 &= \frac{1}{2}h^4 -3ah^4 \\
            \frac{1}{5}h^5 &\neq \frac{1}{2}h^5 -4ah^5
        \end{align*}
        故它具有3阶代数精度。
    \end{itemize}
    \end{solution}
    \item[2.]\begin{solution}
        \begin{itemize}
            \item[(1)] 使用复化梯形公式,
            \begin{equation*}
                \int_{a}^b f(x)\dd x \approx \frac{h}{2}[f(a)+2\sum_{k=1}^{n-1}f(x_k)+f(b)]
            \end{equation*}
            $n=8$,$x_k=\frac{k}{8}(k=0,1,\cdots,8)$,$h=\frac{1}{8}$
            \begin{equation*}
                f(x_k)=\frac{k/8}{4+(k/8)^2}=\frac{8k}{256+k^2}
            \end{equation*}
            \begin{align*}
                \int_{0}^1 \frac{x}{4+x^2}\dd x = \frac{1}{16}\left(\frac{1}{4}+2\sum_{k=1}^{7}\frac{8k}{256+k^2}+\frac{1}{5}\right)=0.12703
            \end{align*}
            使用复化Simpson公式,
            \begin{equation*}
                \int_a^b f(x)\dd x \approx \frac{h}{6}[f(a)+4\sum_{k=0}^{n-1}f(x_{k+\frac{1}{2}})+2\sum_{k=1}^{n-1}f(x_k)+f(b)]
            \end{equation*}
            $x_{k+\frac{1}{2}}=\frac{2k+1}{16}$,
            \begin{equation*}
                f(x_{k+\frac{1}{2}})=\frac{16\times (2k+1)}{256\times 4+(2k+1)^2}=\frac{32k+16}{4k^2+4k+1025}
            \end{equation*}
            有
            \begin{equation*}
                \int_{0}^1 \frac{x}{4+x^2}\dd x = \frac{1}{48}\left(\frac{1}{4}+4\sum_{k=0}^7\frac{32k+16}{4k^2+4k+1025}+2\sum_{k=1}^7\frac{8k}{256+k^2}+\frac{1}{5}\right)=0.11678
            \end{equation*}
        \end{itemize}
    \end{solution}
    \item[4.] \begin{solution}
        使用 Simpson 公式,
        \begin{equation*}
            S = \frac{1}{6}(1+4\frac{1}{\sqrt{\ee}}+\frac{1}{\ee})\approx 0.63233
        \end{equation*}
        误差为
        \begin{equation*}
            R_S = -\frac{1}{180}(\frac{1}{2})^4\ee^{-\eta}=-\frac{1}{2880}\ee^{-\eta}
        \end{equation*}
        其中 $\eta\in(0,1)$,故
        \begin{equation*}
            |R_S|\leq \frac{1}{2880} \approx 3.47 \times 10^{-4}
        \end{equation*}
    \end{solution}
    \item[5.] \begin{solution}
        根据插值型求积公式的余项,
        \begin{align*}
            I_n &= \sum_{k=0}^n f(x_k) \int_a^b l_k(x)\dd x\\
            R[f]&=I-I_n=\int_{a}^b \frac{f^{(n+1)}(\eta)}{(n+1)!}\prod_{i=0}^n (x-x_i)\dd x
        \end{align*}
        有
        \begin{align*}
            \int_a^b f(x)\dd x&=f(a)\int_a^b 1\dd x+\int_a^b f^\prime(\eta)(x-a)\dd x=(b-a)f(a)+\frac{f^\prime(\eta)}{2}(b-a)^2\\
            \int_a^b f(x)\dd x&=f(b)\int_a^b 1\dd x+\int_a^b f^\prime(\eta)(x-b)\dd x=(b-a)f(b)-\frac{f^\prime(\eta)}{2}(b-a)^2
        \end{align*}
        根据泰勒展开式和积分中值定理,$(x-\frac{a+b}{2})^2$ 在 $x\in(a,b)$ 上不变号,有
        \begin{align*}
            \int_a^b f(x)\dd x&=\int_a^b \left[f\left(\frac{a+b}{2}\right)+f^\prime\left(\frac{a+b}{2}\right)\left(x-\frac{a+b}{2}\right)+\frac{f^{\prime\prime}(\eta)}{2}\left(x-\frac{a+b}{2}\right)^2\right]\dd x\\
            &=\left.\left[f\left(\frac{a+b}{2}\right)x+f^\prime\left(\frac{a+b}{2}\right)\left(x-\frac{a+b}{2}\right)^2+\frac{f^{\prime\prime}(\eta)}{6}\left(x-\frac{a+b}{2}\right)^3\right]\right|_a^b\\
            &=(b-a)f\left(\frac{a+b}{2}\right)+\frac{f^{\prime\prime}(\eta)}{24}(b-a)^3
        \end{align*}
    \end{solution}
    \item[7.] \begin{solution}
        复化梯形公式的积分余项为
        \begin{equation*}
            R = -\frac{b-a}{12}h^2f^{\prime\prime}(\eta)
        \end{equation*}
        令 $M=\max_{x\in(a,b)}f^{\prime\prime}(x)$,则
        \begin{equation*}
            |R|\leq \frac{b-a}{12}h^2M \leq \epsilon
        \end{equation*}
        则
        \begin{equation*}
            h=\frac{b-a}{n} \leq \sqrt{\frac{12\epsilon}{(b-a)M}}
        \end{equation*}
        即
        \begin{equation*}
            n\geq \frac{b-a}{2}\sqrt{\frac{(b-a)M}{3\epsilon}}
        \end{equation*}
    \end{solution}
    \item[8.] \begin{solution}
        梯形值递推公式
        \begin{equation*}
            T_0^{(k+1)}=\frac{1}{2}T_0^{(k)}+\frac{h^{(k+1)}}{2}\sum_{i=0}^{2^{k}-1}f(x_{i+\frac{1}{2}})
        \end{equation*}
        以及外推方法
        \begin{equation*}
            T_m^{(k)}=T_{m-1}^{(k+1)}+\frac{1}{4^m-1}(T_{m-1}^{(k+1)}-T_{m-1}^{(k)})
        \end{equation*}
        对$\frac{2}{\sqrt{\pi}}\int_0^1 \ee^{-x}\dd x=\int_0^1 \frac{2}{\sqrt{\pi}}\ee^{-x}\dd x$ 求 Romberg 积分:
        \begin{center}
        \begin{tabular}{cccccc}
            \toprule
            $k$ & $h$ & $T_0^{(k)}$ & $T_1^{(k)}$ & $T_2^{(k)}$ &  $T_3^{(k)}$ \\
            \midrule
            0 & 1 & 0.7717433 \\
            1 & $\frac{1}{2}$ & 0.7280699	& 0.7135122 & 	\\
            2 & $\frac{1}{4}$ & 0.7169828	& 0.7132870	& 0.7132720	& \\
            3 & $\frac{1}{8}$ & 0.7142002	& 0.7132726 & 	0.7132717 &	0.7132717 \\
            \bottomrule
        \end{tabular}
        \end{center}
        可得 $\frac{2}{\sqrt{\pi}}\int_0^1 \ee^{-x}\dd x\approx 0.7132717$。
    \end{solution}
    \item[10.] \begin{proof}
        由于泰勒展开式,
        \begin{equation*}
            \sin x=x-\frac{x^3}{3!}+\frac{x^5}{5!}-\cdots
        \end{equation*}
        故
        \begin{equation*}
            n\sin\frac{\pi}{n}=n(\frac{\pi}{n}-\frac{\pi^3}{3!n^3}+\frac{\pi^5}{5!n^5}-\cdots)=\pi-\frac{\pi^3}{3!n^2}+\frac{\pi^5}{5!n^4}-\cdots
        \end{equation*}
        对于 $f(n)=n\sin\frac{\pi}{n}$,考虑到 Richardson 外推加速方法
        \begin{equation*}
            f_m(2n)=\frac{4^m}{4^m-1}f_{m-1}(n)-\frac{1}{4^m-1}f_{m-1}(2n)
        \end{equation*}
        有
        \begin{center}
            \begin{tabular}{cccc}
                \toprule
                $n$ & $f_0(n)$ & $f_1(n)$ & $f_2(n)$ \\
                \midrule
                3  &  2.598076  \\
                6  &  3.000000   &  3.133975  \\
                12  &  3.105829    &    3.141105    &    3.141580   \\
                \bottomrule
            \end{tabular}
        \end{center}
        得到 $\pi\approx 3.141580$。
    \end{proof}
    \item[11.] \begin{solution}
        \begin{itemize}
            \item[(1)] 使用 Romberg 法得到 $\int_1^3 \frac{\dd y}{y}\approx 1.098612$。
            \begin{center}
            \begin{tabular}{cccccccc}
            \toprule
            $k$ & $h$ & $T_0^{(k)}$ & $T_1^{(k)}$ & $T_2^{(k)}$ &  $T_3^{(k)}$ & $T_4^{(k)}$ & $T_5^{(k)}$ \\
            \midrule
            0  & 2	& 1.333333 \\
            1  & 1  & 1.166667&	1.111111	\\
            2 &	$\frac12$    &  1.116667	&1.100000 &	1.099259	 \\
            3 &	$\frac14$    & 1.103211 &	1.098725	& 1.098640 &	1.098631	\\
            4 &	$\frac18$    & 1.099768	& 1.098620	& 1.098613	& 1.098613	& 1.098613 \\	
            5 &	$\frac1{16}$ &     1.098902	& 1.098613	& 1.098612	& 1.098612	& 1.098612	& 1.098612 \\
            \bottomrule
            \end{tabular}
            \end{center}
            \item[(2)]使用变换 $y=x+2$,有
            \begin{equation*}
                \int_1^3 \frac{\dd y}{y} = \int_{-1}^1 \frac{\dd x}{x+2}
            \end{equation*}
            令 $f(x)=\frac{1}{x+2}$,使用三点 Guass--Legendre 公式,有
            \begin{equation*}
                \int_{-1}^1 \frac{\dd x}{x+2} \approx \frac{5}{9} f\left(-\frac{\sqrt{15}}{5}\right)+\frac{8}{9}f(0)+\frac{5}{9}f\left(\frac{\sqrt{15}}{5}\right)=\frac{56}{51}\approx 1.098039
            \end{equation*}
            使用五点 Guass--Legendre 公式,查表 4.5 有
            \begin{align*}
                \int_{-1}^1 \frac{\dd x}{x+2} \approx& 0.2369269f(-0.9061793)+0.2369269f(0.9061793)\\
                &+0.4786287f(-0.5384693)+0.4786287f(0.5384693)+0.5688889f(0)\\
                \approx& 1.098609
            \end{align*}
            \item[(3)] 区间等分4等份,$[1,1.5],[1.5,2],[2,2.5],[2.5,3]$,复化两点 Guass 公式
            \begin{align*}
                \int_{1}^{1.5} \frac{\dd y}{y}&=\frac{1}{4}\int_{-1}^1 \frac{\dd x}{\frac{1}{4}x+\frac{5}{4}}=\frac{1}{4}\left(\frac{1}{\frac{1}{4}\left(-\frac{1}{\sqrt{3}}\right)+\frac{5}{4}}+\frac{1}{\frac{1}{4}\left(\frac{1}{\sqrt{3}}\right)+\frac{5}{4}}\right)=0.40540541\\
                \int_{1.5}^2 \frac{\dd y}{y}&=\frac{1}{4}\int_{-1}^1 \frac{\dd x}{\frac{1}{4}x+\frac{7}{4}}=\frac{1}{4}\left(\frac{1}{\frac{1}{4}\left(-\frac{1}{\sqrt{3}}\right)+\frac{7}{4}}+\frac{1}{\frac{1}{4}\left(\frac{1}{\sqrt{3}}\right)+\frac{7}{4}}\right)=0.28767123\\
                \int_{2}^{2.5} \frac{\dd y}{y}&=\frac{1}{4}\int_{-1}^1 \frac{\dd x}{\frac{1}{4}x+\frac{9}{4}}=\frac{1}{4}\left(\frac{1}{\frac{1}{4}\left(-\frac{1}{\sqrt{3}}\right)+\frac{9}{4}}+\frac{1}{\frac{1}{4}\left(\frac{1}{\sqrt{3}}\right)+\frac{9}{4}}\right)=0.22314050\\
                \int_{2.5}^{3} \frac{\dd y}{y}&=\frac{1}{4}\int_{-1}^1 \frac{\dd x}{\frac{1}{4}x+\frac{11}{4}}=\frac{1}{4}\left(\frac{1}{\frac{1}{4}\left(-\frac{1}{\sqrt{3}}\right)+\frac{11}{4}}+\frac{1}{\frac{1}{4}\left(\frac{1}{\sqrt{3}}\right)+\frac{11}{4}}\right)=0.18232044\\
            \end{align*}
            故
            \begin{equation*}
                \int_{1}^{3} \frac{\dd y}{y} = \int_{1}^{1.5} \frac{\dd y}{y} + \int_{1.5}^{2} \frac{\dd y}{y} + \int_{2}^{2.5} \frac{\dd y}{y}+ \int_{2.5}^{3} \frac{\dd y}{y}=1.098538
            \end{equation*}
        \end{itemize}
        真实值为
        \begin{equation*}
            \int_1^3 \frac{\dd y}{y} = \ln(y)\bigg|_1^3 = 1.098612
        \end{equation*}
        可以看到 Romberg 法结果误差最小。
    \end{solution}
    \item[补充1.] \begin{solution}
        权函数 $\frac{1}{\sqrt{x}}$,设正交多项式零点为 $x_0$ 和 $x_1$,有正交多项式 $w(x)=(x-x_0)(x-x_1)=x^2+bx+c$ 满足与 1 和 $x$ 的带权正交
        \begin{align*}
            \int_0^1 \frac{1}{\sqrt{x}}w(x)\dd x&=\int_0^1 x^{\frac{3}{2}}+bx^{\frac{1}{2}}+cx^{-\frac{1}{2}}\dd x=\frac{2}{5}+\frac{2}{3}b+2c=0\\
            \int_0^1 \frac{1}{\sqrt{x}}xw(x)\dd x&=\int_0^1 x^{\frac{5}{2}}+bx^{\frac{3}{2}}+cx^{\frac{1}{2}}\dd x=\frac{2}{7}+\frac{2}{5}b+\frac{2}{3}c=0
        \end{align*}
        解得
        \begin{equation*}
            b =-\frac{7}{6} \quad c = \frac{3}{35}
        \end{equation*}
        故 $w(x)=x^2-\frac{7}{6}x+\frac{3}{35}=0$ 解得
        \begin{equation*}
            x_0=\frac{1}{7}\left(3-2\sqrt{\frac{6}{5}}\right)=0.115587\quad x_1=\frac{1}{7}\left(3+2\sqrt{\frac{6}{5}}\right)=0.741556
        \end{equation*}
        考虑
        \begin{equation*}
            \int_{0}^1 \frac{1}{\sqrt{x}}f(x)\dd x\approx A_0f(x_0)+A_1f(x_1)
        \end{equation*}
        对 $f(x)=1,x$ 都准确成立,有
        \begin{align*}
            2 &= A_0 + A_1 \\
            \frac{2}{3} &= A_0x_0+A_1x_1
        \end{align*}
        解得
        \begin{equation*}
            A_0=\frac{2-6x_1}{3x_0-3x_1}=1.304290 \quad A_1=\frac{6x_0-2}{3x_0-3x_1}=0.695710
        \end{equation*}
        所以构造出的 Guass 积分公式为
        \begin{equation*}
            \int_{0}^1 \frac{1}{\sqrt{x}}f(x)\dd x\approx 1.304290f(0.115587)+0.695710f(0.741556)
        \end{equation*}
    \end{solution}
    \item[补充2.] \begin{solution}
        \begin{itemize}
            \item[(1)] 由于是等分节点,所以三点插值型积分公式即 Simpson 公式,
            \begin{align*}
                I_n &= \frac{x_4-x_0}{6}[f(x_0)+4f(x_2)+f(x_4)] \\
                    &= \frac{1.4-1.0}{6}\times[0.2500+4\times 0.2066+0.1736]\\
                    &= 0.0833333
            \end{align*}
            五点插值型积分公式即 Cotes 公式,
            \begin{align*}
                I_n &=\frac{x_4-x_0}{90}[7f(x_0)+32f(x_1)+12f(x_2)+32f(x_3)+7f(x_4)]\\
                &=\frac{1.4-1.0}{90}\times [7\times 0.2500+32\times 0.2268+12\times 0.2066+32\times 0.1890+7\times 0.1736]\\
                &=0.0833333
            \end{align*}
            \item[(2)] 复化梯形公式求解,
            \begin{align*}
                T_n&=\frac{0.1}{2}[f(1.0)+2f(1.1)+2f(1.2)+2f(1.3)+f(1.4)]\\
                &=\frac{0.1}{2}\times (0.2500+2\times 0.2268+2\times 0.2066+2\times 0.1890+0.1736)\\
                &=0.0834200
            \end{align*}
            \item[(3)] 使用复化 Simpson 公式求解,
            \begin{align*}
                S_n&=\frac{0.2}{6}[f(1.0)+4f(1.1)+f(1.2)+f(1.2)+4f(1.3)+f(1.4)]\\
                &=\frac{0.2}{6}\times [0.2500+4\times 0.2268+2\times 0.2066+4\times 0.1890+0.1736]\\
                &=0.0833333
            \end{align*}
        \end{itemize}
    \end{solution}
\end{itemize}

\end{document}