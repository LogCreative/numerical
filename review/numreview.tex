\documentclass[twocolumn]{ctexart}
\pagestyle{plain}
\ctexset{paragraph/beforeskip=0pt,subparagraph/beforeskip=0pt}
\usepackage{amsmath}
\usepackage{amssymb}
\usepackage{graphicx}
\usepackage{geometry}
\geometry{left=1cm,right=1cm,top=2cm,bottom=2cm}
\def\ee{\mathrm{e}}
\def\dd{\mathrm{d}}
\def\jj{\mathrm{j}}
\title{计算方法复习}
\date{}
\begin{document}
\maketitle

% *表示不重要

\section{绪论}
误差、有效数字、映射误差

\section{方程求根}
迭代法:收敛性(全局收敛:$C^1[a,b]$、$R\subseteq I$、映射收缩$|\phi^\prime| <1$;局部收敛:$C^1(x^*)$、$|\phi^\prime(x^*)|<1$)、收敛速度(整数阶收敛,$\phi^{(p)}(x^*)\neq 0$,特殊$0\leq \phi^{\prime}(x^*)<1$)

Newton 迭代法、弦截法、割线法

\section{方程组求解}

直接法:顺序 Guass 消元、列主元消元、Dolittle 分解(充要条件:能够通过顺序 Guass 消元过程化为上三角阵,顺序主子式$1\sim n-1$非零,LU,LDU)、误差分析(舍入误差)、向量范数、矩阵范数、条件数、\textbf{谱半径}

迭代法:Guass-Seodel 迭代、Jacobi 迭代、收敛性(充分:严格对角占优,行(列)范数<1;充要:谱半径<1)、松弛迭代*(必要:$0<\omega<2$,对称正定阵时充要)、收敛速度(与谱半径有关)

\section{插值与逼近}

插值:Lagrange 插值、Newton 插值($\Rightarrow$ 唯一)、\textbf{插值余项}、Runge 现象(可画图,解决方法:分段,分段三次 Hermite 插值*、分段三次样条*,余项)

逼近:$\lVert\cdot \rVert_2$下的逼近(数据、函数下的逼近,投影,法方程)、函数的范数与内积、正交多项式(Legdendre多项式 $[-1,1],\rho(x)=1$,Chebyshev 多项式 $[-1,1],\rho(x)=\frac{1}{\sqrt{1-x^2}}$,最多记到两阶,最佳平方逼近)

\section{数值积分}

插值型:等分节点(Newton-Cotes,梯形公式、Simpson公式、复化梯形公式、复化 Simpson 公式、阶数、积分余项、加速)、不等分节点(代数精度、高斯型)

\section{常微分方程数值解}

Euler 公式、后退 Euler 公式、梯形公式、改进 Euler 公式、Euler 两步格式(5个)

截断误差分析(Runge-Kutta*、Taylor展开)、稳定性分析(针对模型问题 $y^\prime=\lambda y,\text{Re}(\lambda)<0$)。

\section{特征值}

幂法(规范化)、反幂法(逆或LU解方程)

QR分解(迭代法或Gram-Schmidt正交化(需要对角线都是正的)都可以,对称与否,分解出是否对角)

\end{document}