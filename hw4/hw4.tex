\documentclass{sjtuarticle}
\allowdisplaybreaks[1]
\usepackage{ntheorem}
\usepackage{float}
\usepackage{bm}
\usepackage{subcaption}
\usepackage{physics}
\usepackage[colorlinks]{hyperref}
\title{作业4}
\author{李子龙\\123033910195}
\begin{document}
\maketitle
% P217: 1,8,9,11,13,14,19

\begin{itemize}
    \item[1.]\begin{solution}
    \begin{align*}
        \bm{A}&=\begin{pmatrix}
            5 & 2 & 1 \\
            -1 & 4 & 2 \\
            2 & -3 & 10
        \end{pmatrix}=\begin{pmatrix}
            5\\ & 4\\ & & 10 
        \end{pmatrix}-\begin{pmatrix}
            \\ 1 \\ -2 & 3 &
        \end{pmatrix}-\begin{pmatrix}
            & -2 & -1 \\ & & -2 \\ &
        \end{pmatrix}=\bm{D}-\bm{L}-\bm{U} \\
        \bm{b}&=\begin{pmatrix}
            -12 \\ 20 \\ 3
        \end{pmatrix}
    \end{align*}
    \begin{itemize}
        \item[(1)] 由于 $\bm{A}$ 是严格占优矩阵,所以 Jacobi 迭代法和 Guass--Seidel 迭代法解这个方程组都是收敛的。

        \item[(2)] 
        \begin{description}
            \item[Jacobi 迭代法] 
        对于 Jacobi 迭代法,其迭代矩阵
        \begin{align*}
            \bm{B}&=\bm{D}^{-1}(\bm{L}+\bm{U})=\begin{pmatrix}
                & -\frac{2}{5} & -\frac{1}{5} \\
                \frac{1}{4} & & -\frac{1}{2} \\
                -\frac{1}{5} & \frac{3}{10} & 
            \end{pmatrix} & \bm{f}&=\bm{D}^{-1}\bm{b}=\begin{pmatrix}
                -\frac{12}{5} \\ 5 \\ \frac{3}{10}
            \end{pmatrix}
        \end{align*}
        取迭代初值 $\bm{x}^{(0)}=(0,0,0)^\top$,记 $\bm{\epsilon}^{(k)}=\bm{x}^{(k)}-\bm{x}^{(k-1)}$,有
        \begin{align*}
            \bm{x}^{(1)}&=\bm{B}\bm{x}^{(0)}+\bm{f}=\left(-2.4, 5, 0.3\right)^\top & \lVert\bm{\epsilon}^{(1)}\rVert_{\infty}&=5. \\
            \bm{x}^{(2)}&=\bm{B}\bm{x}^{(1)}+\bm{f}=\left(-4.46, 4.25, 2.28\right)^\top & \lVert\bm{\epsilon}^{(2)}\rVert_{\infty}&=2.06 \\
            \bm{x}^{(3)}&=\bm{B}\bm{x}^{(2)}+\bm{f}=\left(-4.556, 2.745, 2.467\right)^\top & \lVert\bm{\epsilon}^{(3)}\rVert_{\infty}&=1.505 \\
            \bm{x}^{(4)}&=\bm{B}\bm{x}^{(3)}+\bm{f}=\left(-3.9914, 2.6275, 2.0347\right)^\top & \lVert\bm{\epsilon}^{(4)}\rVert_{\infty}&=0.5646 \\
            \bm{x}^{(5)}&=\bm{B}\bm{x}^{(4)}+\bm{f}=\left(-3.85794, 2.9848, 1.88653\right)^\top & \lVert\bm{\epsilon}^{(5)}\rVert_{\infty}&=0.3573 \\
            \bm{x}^{(6)}&=\bm{B}\bm{x}^{(5)}+\bm{f}=\left(-3.971226, 3.09225, 1.967028\right)^\top & \lVert\bm{\epsilon}^{(6)}\rVert_{\infty}&=0.113286 \\
            \bm{x}^{(7)}&=\bm{B}\bm{x}^{(6)}+\bm{f}=\left(-4.0303056, 3.0236795, 2.0219202\right)^\top & \lVert\bm{\epsilon}^{(7)}\rVert_{\infty}&=0.0685705 \\
            \bm{x}^{(8)}&=\bm{B}\bm{x}^{(7)}+\bm{f}=\left(-4.01385584, 2.9814635, 2.01316497\right)^\top & \lVert\bm{\epsilon}^{(8)}\rVert_{\infty}&=0.042216 \\
            \bm{x}^{(9)}&=\bm{B}\bm{x}^{(8)}+\bm{f}=\left(-3.99521839, 2.98995356, 1.99721022\right)^\top & \lVert\bm{\epsilon}^{(9)}\rVert_{\infty}&=0.01863745 \\
            \bm{x}^{(10)}&=\bm{B}\bm{x}^{(9)}+\bm{f}=\left(-3.99542347, 3.00259029, 1.99602975\right)^\top & \lVert\bm{\epsilon}^{(10)}\rVert_{\infty}&=0.01263674 \\
            \bm{x}^{(11)}&=\bm{B}\bm{x}^{(10)}+\bm{f}=\left(-4.00024207, 3.00312926, 1.99986178\right)^\top & \lVert\bm{\epsilon}^{(11)}\rVert_{\infty}&=0.0048186 \\
            \bm{x}^{(12)}&=\bm{B}\bm{x}^{(11)}+\bm{f}=\left(-4.00122406, 3.00000859, 2.00098719\right)^\top & \lVert\bm{\epsilon}^{(12)}\rVert_{\infty}&=0.00312067 \\
            \bm{x}^{(13)}&=\bm{B}\bm{x}^{(12)}+\bm{f}=\left(-4.00020088, 2.99920039, 2.00024739\right)^\top & \lVert\bm{\epsilon}^{(13)}\rVert_{\infty}&=0.00102319 \\
            \bm{x}^{(14)}&=\bm{B}\bm{x}^{(13)}+\bm{f}=\left(-3.99972963, 2.99982609, 1.99980029\right)^\top & \lVert\bm{\epsilon}^{(14)}\rVert_{\infty}&=0.0006257 \\
            \bm{x}^{(15)}&=\bm{B}\bm{x}^{(14)}+\bm{f}=\left(-3.99989049, 3.00016745, 1.99989375\right)^\top & \lVert\bm{\epsilon}^{(15)}\rVert_{\infty}&=0.00034136 \\
            \bm{x}^{(16)}&=\bm{B}\bm{x}^{(15)}+\bm{f}=\left(-4.00004573, 3.0000805, 2.00002833\right)^\top & \lVert\bm{\epsilon}^{(16)}\rVert_{\infty}&=0.00015524 \\
            \bm{x}^{(17)}&=\bm{B}\bm{x}^{(16)}+\bm{f}=\left(-4.00003787, 2.9999744, 2.0000333, \right)^\top & \lVert\bm{\epsilon}^{(17)}\rVert_{\infty}&=0.0001061 \\
            \bm{x}^{(18)}&=\bm{B}\bm{x}^{(17)}+\bm{f}=\left(-3.99999642, 2.99997389, 1.99999989\right)^\top & \lVert\bm{\epsilon}^{(18)}\rVert_{\infty}&=4.14468074\times 10^{-5}
        \end{align*}
        总共需要 18 次满足要求。
        \item[Guass--Seidel迭代法] 对于Guass--Seidel迭代法,
        \begin{align*}
            \bm{G}&=(\bm{D}-\bm{L})^{-1}\bm{U}=\begin{pmatrix}
                5 &  &  \\
                -1 & 4 &  \\
                2 & -3 & 10
            \end{pmatrix}^{-1}\begin{pmatrix}
                & -2 & -1 \\ & & -2 \\ &
            \end{pmatrix}
            % =\begin{pmatrix}
            %     0.2 \\
            %     0.05 & 0.25 & \\
            %     -0.025 & 0.075 & 0.1
            % \end{pmatrix}\begin{pmatrix}
            %     & -2 & -1 \\ & & -2 \\ &
            % \end{pmatrix}
            =\begin{pmatrix}
                0 & -0.4 & -0.2 \\
                0 & -0.1 & -0.55 \\
                0 & 0.05 & -0.125
            \end{pmatrix}\\
            \bm{f}&=(\bm{D}-\bm{L})^{-1}\bm{b}=\begin{pmatrix}
                5 &  &  \\
                -1 & 4 &  \\
                2 & -3 & 10
            \end{pmatrix}^{-1}\begin{pmatrix}
                -12 \\ 20 \\ 3
            \end{pmatrix}=\begin{pmatrix}
                -2.4 \\ 4.4 \\ 2.1
            \end{pmatrix}
        \end{align*}
        取迭代初值 $\bm{x}^{(0)}=(0,0,0)^\top$,记 $\bm{\epsilon}^{(k)}=\bm{x}^{(k)}-\bm{x}^{(k-1)}$,有
        \begin{align*}
            \bm{x}^{(1)}&=\bm{G}\bm{x}^{(0)}+\bm{f}=\left(-2.4, 4.4, 2.1\right)^\top & \lVert\bm{\epsilon}^{(1)}\rVert_{\infty}&=4.4 \\
\bm{x}^{(2)}&=\bm{G}\bm{x}^{(1)}+\bm{f}=\left(-4.58, 2.805, 2.0575\right)^\top & \lVert\bm{\epsilon}^{(2)}\rVert_{\infty}&=2.18 \\
\bm{x}^{(3)}&=\bm{G}\bm{x}^{(2)}+\bm{f}=\left(-3.9335, 2.987875, 1.9830625\right)^\top & \lVert\bm{\epsilon}^{(3)}\rVert_{\infty}&=0.6465 \\
\bm{x}^{(4)}&=\bm{G}\bm{x}^{(3)}+\bm{f}=\left(-3.9917625, 3.01052813, 2.00151094\right)^\top & \lVert\bm{\epsilon}^{(4)}\rVert_{\infty}&=0.0582625 \\
\bm{x}^{(5)}&=\bm{G}\bm{x}^{(4)}+\bm{f}=\left(-4.00451344, 2.99811617, 2.00033754\right)^\top & \lVert\bm{\epsilon}^{(5)}\rVert_{\infty}&=0.01275094 \\
\bm{x}^{(6)}&=\bm{G}\bm{x}^{(5)}+\bm{f}=\left(-3.99931398, 3.00000274, 1.99986362\right)^\top & \lVert\bm{\epsilon}^{(6)}\rVert_{\infty}&=0.00519946 \\
\bm{x}^{(7)}&=\bm{G}\bm{x}^{(6)}+\bm{f}=\left(-3.99997382, 3.00007474, 2.00001718\right)^\top & \lVert\bm{\epsilon}^{(7)}\rVert_{\infty}&=0.00065984 \\
\bm{x}^{(8)}&=\bm{G}\bm{x}^{(7)}+\bm{f}=\left(-4.00003333, 2.99998307, 2.00000159\right)^\top & \lVert\bm{\epsilon}^{(8)}\rVert_{\infty}&=9.16628308\times 10^{-5}
        \end{align*}
        总共需要 8 次满足要求。
    \end{description}
    \end{itemize}
    \end{solution}
    \item[8.]\begin{solution}
        \begin{align*}
            \bm{A}&=\begin{pmatrix}
                1 & 0 & -\frac14 & -\frac14 \\
                0 & 1 & -\frac14 & -\frac14 \\
                -\frac14 & -\frac14 & 1 & 0 \\
                -\frac14 & -\frac14 & 0 & 1
            \end{pmatrix}=\begin{pmatrix}
                1\\
                &1\\
                &&1\\
                &&&1
            \end{pmatrix}-\begin{pmatrix}
                \\
                \\
                \frac14 & \frac14 \\
                \frac14 & \frac14 & &
            \end{pmatrix}-\begin{pmatrix}
                &&\frac14 & \frac14 \\
                &&\frac14 & \frac14 \\
                \\
                \\
            \end{pmatrix}=\bm{D}-\bm{L}-\bm{U}
            \\ \bm{b}&=\begin{pmatrix}
                \frac12 \\ \frac12 \\ \frac12 \\ \frac12
            \end{pmatrix}
        \end{align*}
        \begin{itemize}
            \item[(1)] 对于 Jacobi 迭代法,
            \begin{equation*}
                \bm{B}_0=\bm{D}^{-1}(\bm{L}+\bm{U})=\begin{pmatrix}
                    1\\
                    &1\\
                    &&1\\
                    &&&1
                \end{pmatrix}^{-1}\begin{pmatrix}
                    && \frac14 & \frac14 \\
                    && \frac14 & \frac14 \\
                    \frac14 & \frac14 \\
                    \frac14 & \frac14 & &
                \end{pmatrix}=\begin{pmatrix}
                    && \frac14 & \frac14 \\
                    && \frac14 & \frac14 \\
                    \frac14 & \frac14 \\
                    \frac14 & \frac14 & &
                \end{pmatrix}
            \end{equation*}
            求它的特征值
            \begin{equation*}
                0=|\lambda \bm{E}-\bm{B}_0|=\left|\begin{matrix}
                    \lambda && -\frac14 & -\frac14 \\
                    & \lambda & -\frac14 & -\frac14 \\
                    -\frac14 & -\frac14 & \lambda\\
                    -\frac14 & -\frac14 & & \lambda
                \end{matrix}\right|=\lambda^2\left(\lambda-\frac{1}{2}\right)\left(\lambda+\frac{1}{2}\right)
            \end{equation*}
            得到 $\lambda_{1,2}=0,\lambda_3=\frac12,\lambda_4=-\frac12$,故谱半径 $\rho(\bm{B}_0)=\frac12$。
            \item[(2)] 对于Guass--Seidel迭代法,
            \begin{equation*}
                \bm{B}_0=(\bm{D}-\bm{L})^{-1}\bm{U}=\begin{pmatrix}
                    1 &  & & \\
                    0 & 1 & & \\
                    -\frac14 & -\frac14 & 1 &  \\
                    -\frac14 & -\frac14 & 0 & 1
                \end{pmatrix}^{-1}\begin{pmatrix}
                    &&\frac14 & \frac14 \\
                    &&\frac14 & \frac14 \\
                    \\
                    \\
                \end{pmatrix}=\begin{pmatrix}
                    && \frac14 & \frac14 \\
                    && \frac14 & \frac14 \\
                    && \frac18 & \frac18 \\
                    && \frac18 & \frac18
                \end{pmatrix}
            \end{equation*}
            求它的特征值
            \begin{equation*}
                0=|\lambda \bm{E}-\bm{B}_0|=\left|\begin{matrix}
                    \lambda && -\frac14 & -\frac14 \\
                    & \lambda & -\frac14 & -\frac14 \\
                    && \lambda - \frac18 & -\frac18 \\
                    && -\frac18 & \lambda - \frac18
                \end{matrix}\right|=\lambda^3\left(\lambda-\frac14\right)
            \end{equation*}
            得到 $\lambda_{1,2,3}=0,\lambda_4=\frac14$,故谱半径 $\rho(\bm{B}_0)=\frac14$。
            \item[(3)] 由于谱半径都小于1,所以 Jacobi 迭代法和 Guass--Seidel 迭代法均收敛。
        \end{itemize}
    \end{solution}
    \item[9.] \begin{solution}
        \begin{align*}
            \bm{A}&=\begin{pmatrix}
                4 & -1 & \\
                -1 & 4 & -1 \\
                   & -1 & 4
            \end{pmatrix}=\bm{D}-\bm{L}-\bm{U}=\begin{pmatrix}
                4\\ & 4 \\ & & 4
            \end{pmatrix}-\begin{pmatrix}
                \\ 1 \\ & 1 & 
            \end{pmatrix}-\begin{pmatrix}
                & 1 & \\ & & 1 \\ & & 
            \end{pmatrix} \\
            \bm{b}&=\begin{pmatrix}
                1\\4\\-3
            \end{pmatrix}
        \end{align*}
        对于 SOR 迭代法,松弛因子为 $\omega$,有
        \begin{align*}
            \bm{L}_\omega &= (\bm{D}-\omega\bm{L})^{-1}[(1-\omega)\bm{D}+\omega\bm{U}]=\begin{pmatrix}
                4 \\ -\omega & 4 \\ & -\omega & 4
            \end{pmatrix}^{-1}\begin{pmatrix}
                4(1-\omega) & \omega \\
                & 4(1-\omega) & \omega \\
                & & 4(1-\omega)
            \end{pmatrix} \\
            \bm{f}&=\omega(\bm{D}-\omega\bm{L})^{-1}\bm{b}=\omega\begin{pmatrix}
                4 \\ -\omega & 4 \\ & -\omega & 4
            \end{pmatrix}^{-1}\begin{pmatrix}
                1\\4\\-3
            \end{pmatrix}
        \end{align*}

        取迭代初值 $\bm{x}^{(0)}=(0,0,0)^\top$,记 $\bm{\epsilon}^{(k)}=\bm{x}^{*}-\bm{x}^{(k)}$,有
        \begin{enumerate}
            \item[(1)]$\omega=1.03$ 需要 5 次迭代:
            \begin{align*}
    \bm{x}^{(1)}&=\bm{L}_{\omega}\bm{x}^{(0)}+\bm{f}=\left(0.2575, 1.09630625, -0.49020114\right)^\top & \lVert\bm{\epsilon}^{(1)}\rVert_{\infty}&=0.2425 \\
\bm{x}^{(2)}&=\bm{L}_{\omega}\bm{x}^{(1)}+\bm{f}=\left(0.53207386, 1.00789304, -0.49826151\right)^\top & \lVert\bm{\epsilon}^{(2)}\rVert_{\infty}&=0.03207386 \\
\bm{x}^{(3)}&=\bm{L}_{\omega}\bm{x}^{(2)}+\bm{f}=\left(0.50107024, 1.00048646, -0.49992689\right)^\top & \lVert\bm{\epsilon}^{(3)}\rVert_{\infty}&=0.00107024 \\
\bm{x}^{(4)}&=\bm{L}_{\omega}\bm{x}^{(3)}+\bm{f}=\left(0.50009316, 1.00002822, -0.49999493\right)^\top & \lVert\bm{\epsilon}^{(4)}\rVert_{\infty}&=9.31555814\times 10^{-5} \\
\bm{x}^{(5)}&=\bm{L}_{\omega}\bm{x}^{(4)}+\bm{f}=\left(0.50000447, 1.00000161, -0.49999974\right)^\top & \lVert\bm{\epsilon}^{(5)}\rVert_{\infty}&=4.47176785\times 10^{-6} 
            \end{align*}
            \item[(2)] $\omega=1$ 需要 6 次迭代:
\begin{align*}
\bm{x}^{(1)}&=\bm{L}_{\omega}\bm{x}^{(0)}+\bm{f}=\left(0.25, 1.0625, -0.484375\right)^\top & \lVert\bm{\epsilon}^{(1)}\rVert_{\infty}&=0.25 \\
\bm{x}^{(2)}&=\bm{L}_{\omega}\bm{x}^{(1)}+\bm{f}=\left(0.515625, 1.0078125, -0.49804688\right)^\top & \lVert\bm{\epsilon}^{(2)}\rVert_{\infty}&=0.015625 \\
\bm{x}^{(3)}&=\bm{L}_{\omega}\bm{x}^{(2)}+\bm{f}=\left(0.50195312, 1.00097656, -0.49975586\right)^\top & \lVert\bm{\epsilon}^{(3)}\rVert_{\infty}&=0.00195312 \\
\bm{x}^{(4)}&=\bm{L}_{\omega}\bm{x}^{(3)}+\bm{f}=\left(0.50024414, 1.00012207, -0.49996948\right)^\top & \lVert\bm{\epsilon}^{(4)}\rVert_{\infty}&=0.00024414 \\
\bm{x}^{(5)}&=\bm{L}_{\omega}\bm{x}^{(4)}+\bm{f}=\left(0.50003052, 1.00001526, -0.49999619\right)^\top & \lVert\bm{\epsilon}^{(5)}\rVert_{\infty}&=3.05175781\times 10^{-5} \\
\bm{x}^{(6)}&=\bm{L}_{\omega}\bm{x}^{(5)}+\bm{f}=\left(0.50000381, 1.00000191, -0.49999952\right)^\top & \lVert\bm{\epsilon}^{(6)}\rVert_{\infty}&=3.81469727\times 10^{-6}
            \end{align*}
            \item[(3)] $\omega=1.1$ 需要 6 次迭代:
            \begin{align*}
\bm{x}^{(1)}&=\bm{L}_{\omega}\bm{x}^{(0)}+\bm{f}=\left(0.275, 1.175625, -0.50170313\right)^\top & \lVert\bm{\epsilon}^{(1)}\rVert_{\infty}&=0.225 \\
\bm{x}^{(2)}&=\bm{L}_{\omega}\bm{x}^{(1)}+\bm{f}=\left(0.57079688, 1.00143828, -0.49943416\right)^\top & \lVert\bm{\epsilon}^{(2)}\rVert_{\infty}&=0.07079688 \\
\bm{x}^{(3)}&=\bm{L}_{\omega}\bm{x}^{(2)}+\bm{f}=\left(0.49331584, 0.99817363, -0.50055883\right)^\top & \lVert\bm{\epsilon}^{(3)}\rVert_{\infty}&=0.00668416 \\
\bm{x}^{(4)}&=\bm{L}_{\omega}\bm{x}^{(3)}+\bm{f}=\left(0.50016617, 1.00007465, -0.49992359\right)^\top & \lVert\bm{\epsilon}^{(4)}\rVert_{\infty}&=0.00016617 \\
\bm{x}^{(5)}&=\bm{L}_{\omega}\bm{x}^{(4)}+\bm{f}=\left(0.50000391, 1.00001462, -0.50000362\right)^\top & \lVert\bm{\epsilon}^{(5)}\rVert_{\infty}&=1.46243508\times 10^{-5} \\
\bm{x}^{(6)}&=\bm{L}_{\omega}\bm{x}^{(5)}+\bm{f}=\left(0.50000363, 0.99999854, -0.50000004\right)^\top & \lVert\bm{\epsilon}^{(6)}\rVert_{\infty}&=3.63040468\times 10^{-6} 
            \end{align*}
        \end{enumerate}
    \end{solution}
    \item[11.]\begin{proof}
        迭代公式
        \begin{equation*}
            \bm{x}^{(k+1)}=\bm{x}^{(k)}+\omega(\bm{b}-\bm{A}\bm{x}^{(k)})=(\bm{E}-\omega\bm{A})\bm{x}^{(k)}+\omega\bm{b}
        \end{equation*}
        讨论 $\bm{L}_\omega=\bm{E}-\omega\bm{A}$ 的特征值 $\lambda$,对于任意的 $\bm{y}\in\mathbf{R}^{n}$,有
        \begin{align}
            \bm{L}_\omega\bm{y}&=\lambda\bm{y} \nonumber \\
            (\bm{E}-\omega\bm{A})\bm{y}&=\lambda\bm{y} \nonumber \\
            \bm{y}-\omega\bm{A}\bm{y}&=\lambda\bm{y} \label{eq:one}
        \end{align}
        对于 $\bm{A}$ 的特征值 $0<\alpha\leq\lambda^\prime\leq\beta$,有
        \begin{equation}
            \bm{A}\bm{y}=\lambda^\prime\bm{y} \label{eq:two}
        \end{equation}
        结合式 \eqref{eq:one} 和 \eqref{eq:two} 有
        \begin{equation*}
            (1-\omega\lambda^\prime-\lambda)\bm{y}=\bm{0}
        \end{equation*}
        考虑到对于 $\forall \bm{y}\in \mathbf{R}^n$ 都成立,有
        \begin{equation*}
            \lambda = 1-\omega\lambda^\prime
        \end{equation*}
        当 $0<\omega<\frac{2}{\beta}$ 时,有
        \begin{equation*}
            -1=1-\frac{2}{\beta}\beta<\lambda<1-0=1
        \end{equation*}
        即 $|\lambda|< 1$,则迭代法收敛。
    \end{proof}
    \item[13.] \begin{solution}
        \begin{itemize}
            \item[(1)] 设 $\bm{z}^{(m)}=\begin{pmatrix}
                \bm{z}_1^{(m)} \\ \bm{z}_2^{(m)}
            \end{pmatrix}$,$\bm{b}=\begin{pmatrix}
                \bm{b}_1 \\ \bm{b}_2
            \end{pmatrix}$,则对于迭代方法
            \begin{equation*}
                \bm{A}\bm{z}_1^{(m+1)}=\bm{b}_1-\bm{B}\bm{z}_2^{(m)},\quad \bm{A}\bm{z}_2^{(m+1)}=\bm{b}_2-\bm{B}\bm{z}_1^{(m)}
            \end{equation*}
            等价为
            \begin{align*}
                \begin{pmatrix}\bm{A} \\ & \bm{A}\end{pmatrix}\bm{z}^{(m+1)}&=\bm{b}-\begin{pmatrix}
                     & \bm{B} \\ \bm{B}
                \end{pmatrix}\bm{z}^{(m)}\\
                \bm{z}^{(m+1)}&=-\begin{pmatrix}\bm{A} \\ & \bm{A}\end{pmatrix}^{-1}\begin{pmatrix}
                    & \bm{B} \\ \bm{B}
               \end{pmatrix}\bm{z}^{(m)}+\begin{pmatrix}\bm{A} \\ & \bm{A}\end{pmatrix}^{-1}\bm{b} \\
               &=-\begin{pmatrix}\bm{A}^{-1} \\ & \bm{A}^{-1}\end{pmatrix}\begin{pmatrix}
                & \bm{B} \\ \bm{B}
                \end{pmatrix}\bm{z}^{(m)}+\begin{pmatrix}\bm{A}^{-1} \\ & \bm{A}^{-1}\end{pmatrix}\bm{b} \\
                &= \begin{pmatrix}
                    & -\bm{A}^{-1}\bm{B} \\ -\bm{A}^{-1}\bm{B} 
                \end{pmatrix}\bm{z}^{(m)}+\begin{pmatrix}\bm{A}^{-1} \\ & \bm{A}^{-1}\end{pmatrix}\bm{b} 
            \end{align*}
            对于迭代矩阵,它的特征值为 $\lambda$,则对于 $\forall \bm{z}=\begin{pmatrix}\bm{z}_1\\\bm{z}_2\end{pmatrix}\in \mathbf{R}^{2n}$,有
            \begin{align*}
                \begin{pmatrix}
                    & -\bm{A}^{-1}\bm{B} \\ -\bm{A}^{-1}\bm{B} 
                \end{pmatrix}\begin{pmatrix}\bm{z}_1\\\bm{z}_2\end{pmatrix}&=\lambda\begin{pmatrix}\bm{z}_1\\\bm{z}_2\end{pmatrix}\\
                \begin{cases}
                    -\bm{A}^{-1}\bm{B}\bm{z}_2=\lambda \bm{z}_1 \\
                    -\bm{A}^{-1}\bm{B}\bm{z}_1=\lambda \bm{z}_2
                \end{cases}\\
                \Rightarrow \begin{cases}
                    (\bm{A}^{-1}\bm{B})^2 \bm{z}_1=\lambda^2 \bm{z}_1\\
                    (\bm{A}^{-1}\bm{B})^2 \bm{z}_2=\lambda^2 \bm{z}_2\\
                \end{cases}
            \end{align*}
            即 $\rho\begin{pmatrix}
                & -\bm{A}^{-1}\bm{B} \\ -\bm{A}^{-1}\bm{B} 
            \end{pmatrix}=\rho(\bm{A}^{-1}\bm{B})$,迭代法收敛的充要条件即 $\rho(\bm{A}^{-1}\bm{B})<1$。
            \item[(2)] 设 $\bm{z}^{(m)}=\begin{pmatrix}
                \bm{z}_1^{(m)} \\ \bm{z}_2^{(m)}
            \end{pmatrix}$,$\bm{b}=\begin{pmatrix}
                \bm{b}_1 \\ \bm{b}_2
            \end{pmatrix}$,则对于迭代方法
            \begin{equation*}
                \bm{A}\bm{z}_1^{(m+1)}=\bm{b}_1-\bm{B}\bm{z}_2^{(m)},\quad \bm{A}\bm{z}_2^{(m+1)}=\bm{b}_2-\bm{B}\bm{z}_1^{(m+1)}
            \end{equation*}
            等价为
            \begin{align*}
                \begin{pmatrix}\bm{A} & \bm{0}\\ \bm{B} & \bm{A}\end{pmatrix}\bm{z}^{(m+1)}&=\bm{b}-\begin{pmatrix}
                    \bm{0}  & \bm{B} \\ \bm{0} & \bm{0}
                \end{pmatrix}\bm{z}^{(m)} \\
                \bm{z}^{(m+1)}&=-\begin{pmatrix}\bm{A} & \bm{0}\\ \bm{B} & \bm{A}\end{pmatrix}^{-1}\begin{pmatrix}
                    \bm{0}  & \bm{B} \\ \bm{0} & \bm{0}
                \end{pmatrix}\bm{z}^{(m)}+\begin{pmatrix}\bm{A} & \bm{0}\\ \bm{B} & \bm{A}\end{pmatrix}^{-1}\bm{b}\\
                &=-\begin{pmatrix}
                    \bm{A}^{-1} & \bm{0} \\ -\bm{A}^{-1}\bm{B}\bm{A}^{-1} & \bm{A}^{-1}
                \end{pmatrix}\begin{pmatrix}
                    \bm{0}  & \bm{B} \\ \bm{0} & \bm{0}
                \end{pmatrix}\bm{z}^{(m)}+\begin{pmatrix}\bm{A} & \bm{0}\\ \bm{B} & \bm{A}\end{pmatrix}^{-1}\bm{b} \\
                &=\begin{pmatrix}
                    \bm{0} & -\bm{A}^{-1}\bm{B} \\ \bm{0} & \bm{A}^{-1}\bm{B}\bm{A}^{-1}\bm{B}
                \end{pmatrix}\bm{z}^{(m)}+\begin{pmatrix}\bm{A} & \bm{0}\\ \bm{B} & \bm{A}\end{pmatrix}^{-1}\bm{b} 
            \end{align*}
            对于迭代矩阵,它的特征值为 $\lambda$,则对于 $\forall \bm{z}=\begin{pmatrix}\bm{z}_1\\\bm{z}_2\end{pmatrix}\in \mathbf{R}^{2n}$,有
            \begin{align*}
                \begin{pmatrix}
                    \bm{0} & -\bm{A}^{-1}\bm{B} \\ \bm{0} & \bm{A}^{-1}\bm{B}\bm{A}^{-1}\bm{B}
                \end{pmatrix}\begin{pmatrix}\bm{z}_1\\\bm{z}_2\end{pmatrix}&=\lambda\begin{pmatrix}\bm{z}_1\\\bm{z}_2\end{pmatrix}\\
                \begin{cases}
                    -\bm{A}^{-1}\bm{B}\bm{z}_2=\lambda\bm{z}_1\\
                    (\bm{A}^{-1}\bm{B})^2\bm{z}_2=\lambda\bm{z}_2
                \end{cases}\\
                \Rightarrow
                \begin{cases}
                    (\bm{A}^{-1}\bm{B})^2\bm{z}_1=\lambda\bm{z}_1\\
                    (\bm{A}^{-1}\bm{B})^2\bm{z}_2=\lambda\bm{z}_2
                \end{cases}
            \end{align*}
            即 $\rho \begin{pmatrix}
                \bm{0} & -\bm{A}^{-1}\bm{B} \\ \bm{0} & \bm{A}^{-1}\bm{B}\bm{A}^{-1}\bm{B}
            \end{pmatrix}=\rho((\bm{A}^{-1}\bm{B})^2)$,迭代法的充要条件即 $\rho((\bm{A}^{-1}\bm{B})^2)=\rho((\bm{A}^{-1}\bm{B}))^2<1$。
        \end{itemize}
        方法 (1) 的收敛速度为 $R_1=-\ln \rho ((\bm{A}^{-1}\bm{B}))$,方法 (2) 的收敛速度为 $R_2=-2\ln \rho((\bm{A}^{-1}\bm{B}))$,即 $R_2=2R_1$,方法 (2) 的收敛速度是方法 (1) 的 2 倍。
    \end{solution}
    \item[14.] \begin{proof}
        若矩阵 $\bm{A}=\begin{pmatrix}
            1 & a & a \\ a & 1 & a \\ a & a & 1
        \end{pmatrix}$ 是正定的,那么
        \begin{equation*}
            \begin{cases}
                \det \begin{pmatrix}
                    1 & a \\ a & 1
                \end{pmatrix}=1-a^2>0&\Rightarrow -1 < a < 1 \\
                \det \bm{A}= (a-1)^2(2a+1)> 0 &\Rightarrow a>-\frac{1}{2}
            \end{cases}
        \end{equation*}
        即当 $-\frac{1}{2}<a<1$ 时,矩阵 $\bm{A}$ 是正定的。

        考察
        \begin{equation*}
            \bm{A}=\begin{pmatrix}
                1 & a & a \\ a & 1 & a \\ a & a & 1
            \end{pmatrix}=\bm{D}-\bm{L}-\bm{U}=\begin{pmatrix}
                1 \\ & 1 \\ & & 1
            \end{pmatrix}-\begin{pmatrix}
                \\ -a \\ -a & -a & 
            \end{pmatrix}-\begin{pmatrix}
                & -a & -a \\ & & -a \\ & &
            \end{pmatrix}
        \end{equation*}
        在 Jacobi 迭代法中,迭代矩阵
        \begin{equation*}
            \bm{B}=\bm{D}^{-1}(\bm{L}+\bm{U})=\begin{pmatrix}
                1 \\ & 1 \\ & & 1
            \end{pmatrix}^{-1}\begin{pmatrix}
                & -a & -a \\
                -a & & -a \\
                -a & -a &
            \end{pmatrix}=\begin{pmatrix}
                & -a & -a \\
                -a & & -a \\
                -a & -a &
            \end{pmatrix}
        \end{equation*}
        考察它的特征值
        \begin{equation*}
            |\lambda \bm{E}-\bm{B}|=0\Rightarrow \left|\begin{matrix}
                \lambda & a & a \\
                a & \lambda & a \\
                a & a & \lambda
            \end{matrix}\right|=(\lambda-a)^2(\lambda+2a)=0\Rightarrow\lambda_{1,2} =a,\lambda_3=-\frac{1}{2a} 
        \end{equation*}
        为了使迭代法收敛,有 $|\lambda|<1$,则需要 $-\frac{1}{2}<a<\frac{1}{2}$。
    \end{proof}
    \item[19.]\begin{proof}
        \begin{itemize}
            \item[(1)] 由于 $(\bm{A}^\top\bm{A})^\top=\bm{A}^\top(\bm{A}^\top)^\top=\bm{A}^\top\bm{A}$,则 $\bm{A}^\top\bm{A}$ 是对称矩阵。
            
            由于 $\bm{A}$ 是非奇异矩阵,对 $\forall \bm{x}\neq 0$,有 $\bm{x}^\top\bm{A}^\top\bm{A}\bm{x}=(\bm{Ax})^\top(\bm{Ax})>0$,所以 $\bm{A}^\top\bm{A}$ 是正定矩阵。

            综上,$\bm{A}^\top\bm{A}$ 是对称正定矩阵。
            \item[(2)] 由于 $\bm{A}^\top\bm{A}$ 是对称矩阵,所以
            \begin{equation}\label{eq:three}
                \text{cond}(\bm{A}^\top\bm{A})_2=\frac{|\lambda_{\max}(\bm{A}^\top\bm{A})|}{|\lambda_{\min}(\bm{A}^\top\bm{A})|}
            \end{equation}
            另一方面,
            \begin{equation}\label{eq:four}
                \text{cond}(\bm{A})_2=\sqrt{\frac{\lambda_{\max}(\bm{A}^\top\bm{A})}{\lambda_{\min}(\bm{A}^\top\bm{A})}}
            \end{equation}
            综合式 \eqref{eq:three} 和 \eqref{eq:four} 有
            \begin{equation*}
                \text{cond}(\bm{A}^\top\bm{A})_2=(\text{cond}(\bm{A})_2)^2
            \end{equation*}
        \end{itemize}
    \end{proof}
\end{itemize}

\end{document}