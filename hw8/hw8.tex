\documentclass{sjtuarticle}
\allowdisplaybreaks[1]
\usepackage{array}
\usepackage{ntheorem}
\usepackage{float}
\usepackage{pgfplots}
\pgfplotsset{compat=newest}
\usepackage{bm}
\usepackage{booktabs}
\usepackage{subcaption}
\usepackage[colorlinks]{hyperref}

\def\dd{\mathrm{d}}
\def\DD{\mathrm{D}}
\def\ee{\mathrm{e}}

\title{作业8}
\author{Log Creative}
\date{2024 年 1 月 3 日}
\begin{document}
\maketitle

% 作业:P140:1, 2,4,5,6,7,8,12,补充题见附件

\begin{itemize}
    \item[1.] \begin{solution}
       \begin{equation*}
            \begin{cases}
                y^\prime = ax+b \\
                y(0) = 0
            \end{cases}
        \end{equation*}
        \textbf{Euler 方法:}\begin{equation*}
            y_{n+1}=y_n+h(ax_n+b)
        \end{equation*}
        取步长 $h=0.1$,计算结果如表所示。
        \begin{table}[h]
            \centering
            \begin{tabular}{ccc|ccc}
                \toprule
                $x_n$ & $y_n$ & $y(x_n)$ & $x_n$ & $y_n$ & $y(x_n)$ \\
                \midrule
                $0.1$   & $0.1 b$       & $0.005 a + 0.1 b$ & $0.6$   & $0.15 a + 0.6 b$      & $0.18 a + 0.6 b$ \\
                $0.2$   & $0.01 a + 0.2 b$      & $0.02 a + 0.2 b$ &                 $0.7$   & $0.21 a + 0.7 b$      & $0.245 a + 0.7 b$ \\
                $0.3$   & $0.03 a + 0.3 b$      & $0.045 a + 0.3 b$ &                 $0.8$   & $0.28 a + 0.8 b$      & $0.32 a + 0.8 b$ \\
                $0.4$   & $0.06 a + 0.4 b$      & $0.08 a + 0.4 b$ &                 $0.9$   & $0.36 a + 0.9 b$      & $0.405 a + 0.9 b$ \\
                $0.5$   & $0.1 a + 0.5 b$       & $0.125 a + 0.5 b$ &                 $1.0$   & $0.45 a + 1.0 b$      & $0.5 a + 1.0 b$ \\
                \bottomrule
            \end{tabular}
        \end{table}

        表达式为
        \begin{equation*}
            y_{n} - y_0 = \sum_{k=0}^{n-1} h(ax_k+b) =\sum_{k=0}^{n-1} h(akh+b) = \sum_{k=0}^{n-1} akh^2+bh = \frac{n(n-1)}{2}ah^2 +nbh
        \end{equation*}
        误差为
        \begin{equation*}
            y(x_n) - y_n = \frac{1}{2}a(nh)^2+b(nh)-\frac{n(n-1)}{2}ah^2 -nbh=\frac{1}{2}nah^2
        \end{equation*}

        \textbf{改进 Euler 方法:}
        \begin{equation*}
            y_{n+1}=y_n+\frac{h}{2}[ax_n+b+a(x_n+h)+b]=y_n+ahx_n+hb+\frac{1}{2}ah^2
        \end{equation*}
        取步长 $h=0.1$,计算结果如表所示。
        \begin{table}[h]
            \centering
            \begin{tabular}{ccc|ccc}
                \toprule
                $x_n$ & $y_n$ & $y(x_n)$ & $x_n$ & $y_n$ & $y(x_n)$ \\
                \midrule
$0.1$   & $0.005 a + 0.1 b$     & $0.005 a + 0.1 b$ & $0.6$   & $0.18 a + 0.6 b$      & $0.18 a + 0.6 b$ \\
$0.2$   & $0.02 a + 0.2 b$      & $0.02 a + 0.2 b$ & $0.7$   & $0.245 a + 0.7 b$     & $0.245 a + 0.7 b$ \\
$0.3$   & $0.045 a + 0.3 b$     & $0.045 a + 0.3 b$ & $0.8$   & $0.32 a + 0.8 b$      & $0.32 a + 0.8 b$ \\
$0.4$   & $0.08 a + 0.4 b$      & $0.08 a + 0.4 b$ & $0.9$   & $0.405 a + 0.9 b$     & $0.405 a + 0.9 b$ \\
$0.5$   & $0.125 a + 0.5 b$     & $0.125 a + 0.5 b$ & $1.0$   & $0.5 a + 1.0 b$       & $0.5 a + 1.0 b$ \\
                \bottomrule
            \end{tabular}
        \end{table}

        表达式为
        \begin{equation*}
            y_n-y_0 =\sum_{k=0}^{n-1}\left(ahx_k + hb + \frac{1}{2}ah^2\right)=\frac{n(n-1)}{2}ah^2 + nbh + \frac{1}{2}ah^2n=\frac{1}{2}an^2h^2+bnh
        \end{equation*}
        误差为
        \begin{equation*}
            y(x_n)-y_n = 0
        \end{equation*}

        可见改进 Euler 方法的结果没有误差,Euler 方法在一次项系数上有一定误差。
    \end{solution}

    \item[2.] \begin{solution}
        改进 Euler 方法的表达式为
        \begin{equation*}
            \begin{aligned}
            y_{n+1}&=y_n + \frac{h}{2}[x_n+y_n+x_n+h+y_n+h(x_n+y_n)]\\
                    &=x_n\left(h+\frac{1}{2}h^2\right) + y_n\left(1+h+\frac{1}{2}h^2\right) + \frac{1}{2}h^2
            \end{aligned}
        \end{equation*}
        取 $h=0.1$,有
        \begin{equation*}
            y_{n+1}=0.105x_n+1.105y_n+0.005
        \end{equation*}
        计算结果如表所示
        \begin{table}[h]
            \centering
            \begin{tabular}{ccc|ccc}
                \toprule
                $x_n$ & $y_n$ & $y(x_n)$ & $x_n$ & $y_n$ & $y(x_n)$ \\
                \midrule
                $0.1$   & $1.1100$      & $1.1103$  &                 $0.6$   & $2.0409$      & $2.0442$ \\
                $0.2$   & $1.2420$      & $1.2428$  &                 $0.7$   & $2.3231$      & $2.3275$ \\
                $0.3$   & $1.3985$      & $1.3997$  &                 $0.8$   & $2.6456$      & $2.6511$ \\
                $0.4$   & $1.5818$      & $1.5836$  &                 $0.9$   & $3.0124$      & $3.0192$ \\
                $0.5$   & $1.7949$      & $1.7974$  &                 $1.0$   & $3.4282$      & $3.4366$ \\
                \bottomrule
            \end{tabular}
        \end{table}

        与精确解比较后可知有2位小数精度。
    \end{solution}

    \item[4.] \begin{proof}
        根据梯形计算格式,
        \begin{align*}
            y_{n+1}&=y_n+\frac{1}{2}[-y_n-y_{n+1}] \\
            y_{n+1}&=\frac{2-h}{2+h}y_n
        \end{align*}
        根据此迭代公式,
        \begin{equation*}
            y_{n}=\left(\frac{2-h}{2+h}\right)^n y_0 = \left(\frac{2-h}{2+h}\right)^n
        \end{equation*}
        当 $h\rightarrow 0$ 时,$x=x_0+nh=nh$,
        \begin{equation*}
            y_{n}=\lim_{h\rightarrow 0}\left(\frac{2-h}{2+h}\right)^n =\lim_{h\rightarrow 0} \left(1-\frac{2h}{2+h}\right)^{\frac{x}{h}}
            =\lim_{h\rightarrow 0} \left(1-\frac{2h}{2+h}\right)^{\frac{2+h}{2h}\frac{2h}{2+h}\frac{x}{h}}=\lim_{h\rightarrow 0}\left[\left(1-\frac{2h}{2+h}\right)^{\frac{2+h}{2h}}\right]^{\frac{2}{2+h}x}=\ee^{-x}
        \end{equation*}
    \end{proof}

    \item[5.] \begin{solution}
        根据微积分基本定理,有
        \begin{align*}
            y=\int_0^x\ee^{t^2}\dd t \Rightarrow
            \begin{cases}
                y^\prime =\ee^{x^2}\\
                y(0)=0
            \end{cases}
        \end{align*}
        构造 Euler 格式,
        \begin{equation*}
            y_{n+1}=y_n + h\ee^{x_n^2}
        \end{equation*}
        取 $h=0.5$,有
        \begin{align*}
            y(0.5)&=0+0.5\times \ee^{0^2}=0.5\\
            y(1)&=0.5+0.5\times \ee^{0.5^2}=1.1420\\
            y(1.5)&=1.1420+0.5\times \ee^{1^2}=2.5011\\
            y(2)&=2.5011+0.5\times \ee^{1.5^2}=7.2450
        \end{align*}
    \end{solution}

    \item[6.] \begin{solution}
        经典的四阶 Runge-Kutta 方法是
        \begin{equation*}
            \begin{cases}
                y_{n+1}=y_n+\frac{h}{6}(K_1+2K_2+2K_3+K_4)\\
                K_1=f(x_n,y_n)\\
                K_2=f(x_n+\frac{h}{2},y_n+\frac{h}{2}K_1)\\
                K_3=f(x_n+\frac{h}{2},y_n+\frac{h}{2}K_2)\\
                K_4=f(x_n+h,y_n+hK_3)
            \end{cases}
        \end{equation*}
        \begin{itemize}
            \item[(1)] \begin{equation*}
                \begin{cases}
                    y^\prime =x+y, & 0<x<1,\\
                    y(0)=1
                \end{cases}
            \end{equation*}
            解得
            \begin{table}[h]
                \centering
                \begin{tabular}{cc}
                    \toprule
                    $x_n$ & $y_n$ \\
                    \midrule
                    0.2 & 1.2428 \\
                    0.4 & 1.5836 \\
                    0.6 & 2.0442 \\
                    0.8 & 2.6510 \\
                    1.0 & 3.4365 \\
                    \bottomrule
                \end{tabular}
            \end{table}

            \item[(2)] \begin{equation*}
                \begin{cases}
                    y^\prime = 3y/(1+x),&0<x<1\\
                    y(0)=1
                \end{cases}
            \end{equation*} 
            解得
            \begin{table}[h]
                \centering
                \begin{tabular}{cc}
                    \toprule
                    $x_n$ & $y_n$ \\
                    \midrule
                    0.2 & 1.7275 \\
                    0.4 & 2.7430 \\
                    0.6 & 4.0942 \\
                    0.8 & 5.8292 \\
                    1.0 & 7.9960 \\
                    \bottomrule
                \end{tabular}
            \end{table}
        \end{itemize}
    \end{solution}

    \item[7.] \begin{proof}
        \begin{equation*}
            \begin{cases}
                y_{n+1}=y_n+\frac{h}{2}(K_2+K_3)\\
                K_1=f(x_n,y_n)\\
                K_2=f(x_n+th,y_h+thK_1)\\
                K_3=f(x_n+(1-t)h,y_n+(1-t)hK_1)
            \end{cases}
        \end{equation*}
        % 将 $K_2,K_3$ 代入第一个式子,
        % \begin{equation*}
        %     y_{n+1}=y_n+\frac{h}{2}[f(x_n+th,y_h+thK_1)+f(x_n+(1-t)h,y_n+(1-t)hK_1)]
        % \end{equation*}
        根据 Taylor 格式,
        \begin{align*}
            K_1&=f_n\\
            K_2&=f_n+th(f_x+f\cdot f_y)_n+\cdots\\
            K_3&=f_n+(1-t)h(f_x+f\cdot f_y)_n+\cdots
        \end{align*}
        代入第一个式子,
        \begin{equation*}
            y_{n+1}=y_n+hf_n+\frac{h^2}{2}(f_x+f\cdot f_y)_n+\cdots
        \end{equation*}
        这与二阶 Taylor 格式的前两阶相同,所以误差是 $O(h^3)$,也就是它是二阶 Runge-Kutta 格式。
    \end{proof}

    \item[8.] \begin{proof}
        对于形如 \begin{equation*}
            \begin{cases}
                y_{n+1}=y_n+h(\lambda_1K_1+\lambda_2K_2+\lambda_3K_3)\\
                K_1=f(x_n,y_n)\\
                K_2=f(x_n+ph,y_n+phK_1)\\
                K_3=f(x_n+qh,y_n+qh(rK_1+sK_2))
            \end{cases}
        \end{equation*}
        的格式,只要满足
        \begin{equation}\label{eq:cond}
            \begin{cases}
                \lambda_2p+\lambda_3q=\frac{1}{2}\\
                \lambda_2p^2+\lambda_3q^2=\frac{1}{3}\\
                \lambda_3pqs=\frac{1}{6}
            \end{cases}
        \end{equation}
        就是三阶 Runge-Kutta 格式。
        % 记微分算子
        % \begin{equation*}
        %     \DD = \frac{\partial}{\partial x}+f\frac{\partial }{\partial y},\quad \DD^2=\frac{\partial^2}{\partial x^2}+2f\frac{\partial^2}{\partial x\partial y}+f^2\frac{\partial^2}{\partial y^2}
        % \end{equation*}
        % 则
        % \begin{equation*}
        %     y^\prime = f, \quad y^{\prime\prime}=\DD f, \quad y^{\prime\prime\prime}=\DD^2f+\frac{\partial f}{\partial y}\DD f
        % \end{equation*}
        % 则三阶 Taylor 格式
        % \begin{equation*}
        %     y_{n+1}=y_n+hf_n+\frac{h^2}{2}\DD f_n+\frac{h^3}{6}(\DD^2 f+f_y\DD f)_n
        % \end{equation*}
        \begin{itemize}
            \item[(1)] \begin{equation*}
                \begin{cases}
                    y_{n+1}=y_n+\frac{h}{4}(K_1+3K_3),\\
                    K_1=f(x_n,y_n),\\
                    K_2=f\left(x_n+\frac{h}{3},y_n+\frac{h}{3}K_1\right),\\
                    K_3=f\left(x_n+\frac{2}{3}h,y_n+\frac{2}{3}hK_2\right)
                \end{cases}
            \end{equation*}
            % \begin{align*}
            %     K_1 &= f_n\\
            %     K_2 &= f_n+\frac{h}{3}\DD f_n+\frac{h^2}{18}\DD^2 f_n+\cdots \\
            %     K_3 &= f_n+\frac{2}{3}h\bar{\DD} f_n + \frac{2}{9}h^2\bar{\DD}^2 f_n+\cdots
            % \end{align*}
            % 这里,
            % \begin{align*}
            %     \bar{\DD}&=\frac{\partial}{\partial x}+K_2\frac{\partial}{\partial y}=\DD + \frac{h}{3}\DD f_n \frac{\partial}{\partial y}+\cdots\\
            %     \bar{\DD}^2&=\DD^2+\cdots
            % \end{align*}
            % 则
            % \begin{equation}
            %     K_3 = f_n+\frac{2}{3}h\DD f_n+\frac{h^2}{2}(\frac{4}{9}\DD^2f+\frac{4}{9}\DD f\cdot f_y)_n
            % \end{equation}
            % 故
            % \begin{equation*}
            %     y_{n+1}=y_n+\frac{h}{4}(f_n+3)
            % \end{equation*}
            这里,
            \begin{equation*}
                \lambda_1=\frac{1}{4},\quad \lambda_2=0,\quad \lambda_3=\frac{3}{4},\quad p=\frac{1}{3},\quad q=\frac{2}{3},\quad r=0,\quad s=1
            \end{equation*}
            容易验证满足式 \eqref{eq:cond}。
            \item[(2)]
            \begin{equation*}
                \begin{cases}
                    y_{n+1}=y_n+\frac{h}{9}(2K_1+3K_2+4K_3),\\
                    K_1=f(x_n,y_n),\\
                    K_2=f\left(x_n+\frac{h}{2},y_n+\frac{h}{2}K_1\right),\\
                    K_3=f\left(x_n+\frac{3}{4}h,y_n+\frac{3}{4}hK_2\right)
                \end{cases}
            \end{equation*}
            这里,
            \begin{equation*}
                \lambda_1=\frac{2}{9},\quad \lambda_2=\frac{1}{3},\quad \lambda_3=\frac{4}{9},\quad p=\frac{1}{2},\quad q=\frac{3}{4},\quad r=0,\quad s=1
            \end{equation*}
            容易验证满足式 \eqref{eq:cond}。
        \end{itemize}
    \end{proof}

    \item[12.] \begin{solution}
        令 $y_1=y,y_2=y^\prime$,
        \begin{itemize}
            \item[(1)]
            \begin{equation*}
                \begin{cases}
                    y_1^\prime=y_2\\
                    y_2^\prime=3y_2-2y
                \end{cases}\xrightarrow{z=y_2}
                \begin{cases}
                    y^\prime =z, y(0)=1\\
                    z^\prime = 3z-2y, z(0)=1
                \end{cases}
            \end{equation*}
            \item[(2)] \begin{equation*}
                \begin{cases}
                    y_1^\prime=y_2\\
                    y_2^\prime=0.1(1-y^2)y_2-y
                \end{cases}\xrightarrow{z=y_2}
                \begin{cases}
                    y^\prime =z,y(0)=1\\
                    z^\prime =0.1(1-y^2)z-y,z(0)=0
                \end{cases}
            \end{equation*}
            \item[(3)] 令 $x_1=x,x_2=x^\prime,r=\sqrt{x^2+y^2}$,\begin{equation*}
                \begin{cases}
                    x_1^\prime=x_2\\
                    x_2^\prime=-\frac{x}{r^3}\\
                    y_1^\prime=y_2\\
                    y_2^\prime=-\frac{y}{r^3}
                \end{cases}
                \xrightarrow{u=x_2,z=y_2}
                \begin{cases}
                    x^\prime=u, x(0)=0.4\\
                    y^\prime=z, y(0)=0\\
                    u^\prime=-\frac{x}{r^3}, u(0)=0\\
                    z^\prime=-\frac{y}{r^3}, z(0)=2
                \end{cases}
            \end{equation*}
        \end{itemize}
    \end{solution}

    \item[补充题.] \begin{solution}
        对于隐式中点公式,
        \begin{equation*}
            y_{n+1}=y_n+hf\left(x_n+\frac{h}{2},\frac{1}{2}(y_n+y_{n+1})\right)
        \end{equation*}
        对于模型方程$y^\prime=\lambda y$,有
        \begin{equation*}
            y_{n+1}=y_n+\frac{1}{2}\lambda h(y_n+y_{n+1})
        \end{equation*}
        也就是
        \begin{equation*}
            y_{n+1}=\frac{1+\frac{1}{2}\lambda h}{1-\frac{1}{2}\lambda h}y_n
        \end{equation*}
        则为了使其绝对稳定,需要满足
        \begin{equation*}
            \left|\frac{1+\frac{1}{2}\lambda h}{1-\frac{1}{2}\lambda h}\right|\leq 1
        \end{equation*}
    \end{solution}
\end{itemize}

\end{document}