\documentclass{sjtuarticle}
\usepackage{ntheorem}
\usepackage{float}
\usepackage{bm}
\usepackage{subcaption}
\usepackage{pgfplots}
\pgfplotsset{compat=newest}
\usepackage[colorlinks]{hyperref}
\title{作业3}
\author{Log Creative}
\date{2023 年 11 月 1 日}
\begin{document}
\maketitle
% P199: 1,2,6,7,9,10,11,13,14,15,18,20,23,29,31,32

\begin{itemize}
    \item[1.] 
% 0.4096x_1+0.1234x_2+0.3678x_3+0.2943x_4=0.4043
% 0.2246x_1+0.3872x_2+0.4015x_3+0.1129x_4=0.1550
% 0.3645x_1+0.1920x_2+0.3781x_3+0.0643x_4=0.4240
% 0.1784x_1+0.4002x_2+0.2786x_3+0.3927x_4=-0.2557
    \begin{solution}
        解方程组,限制计算精度为4位小数。
        \begin{equation*}
            \begin{cases}
                0.4096x_1+0.1234x_2+0.3678x_3+0.2943x_4=0.4043 \\
                0.2246x_1+0.3872x_2+0.4015x_3+0.1129x_4=0.1550 \\
                0.3645x_1+0.1920x_2+0.3781x_3+0.0643x_4=0.4240 \\
                0.1784x_1+0.4002x_2+0.2786x_3+0.3927x_4=-0.2557
            \end{cases}
        \end{equation*}
        \begin{itemize}
            \item[(1)] 高斯消元法:
        \begin{align*}
            \begin{pmatrix}
                \bm{A} & \bm{b}
            \end{pmatrix}=\begin{pmatrix}
                0.4096 &       0.1234 &       0.3678 &       0.2943 &        0.4043\\
                0.2246 &       0.3872 &       0.4015 &       0.1129 &        0.1550\\
                0.3645 &       0.1920 &       0.3781 &       0.0643 &        0.4240\\
                0.1784 &       0.4002 &       0.2786 &       0.3927 &       -0.2557\\
            \end{pmatrix}\rightarrow
            \begin{pmatrix}
                0.4096 &       0.1234 &       0.3678 &       0.2943 &        0.4043\\
                0.0000 &       0.3195 &       0.1998 &      -0.0485 &       -0.0667\\
                0.0000 &       0.0822 &       0.0508 &      -0.1976 &        0.0642\\
                0.0000 &       0.3465 &       0.1184 &       0.2645 &       -0.4318\\
            \end{pmatrix} \\
            \rightarrow\begin{pmatrix}
            0.4096 &       0.1234 &       0.3678 &       0.2943 &        0.4043\\
            0.0000 &       0.3195 &       0.1998 &      -0.0485 &       -0.0667\\
            0.0000 &       0.0000 &      -0.0006 &      -0.1851 &        0.0814\\
            0.0000 &       0.0000 &      -0.0983 &       0.3171 &       -0.3595\\
            \end{pmatrix}\rightarrow
            \begin{pmatrix}
            0.4096 &       0.1234 &       0.3678 &       0.2943 &        0.4043\\
            0.0000 &       0.3195 &       0.1998 &      -0.0485 &       -0.0667\\
            0.0000 &       0.0000 &      -0.0006 &      -0.1851 &        0.0814\\
            0.0000 &       0.0000 &       0.0000 &      30.6427 &      -13.6955\\
            \end{pmatrix}
        \end{align*}
        回代解得
        \begin{align*}
            x_1&=     -0.1800 & x_2&=     -1.6617 & x_3&=      2.2148 & x_4&=     -0.4469
        \end{align*}
        \item[(2)] 列主元消元法:
        \begin{align*}
            \begin{pmatrix}
                \bm{A} & \bm{b}
            \end{pmatrix}=
            \begin{pmatrix}
                0.4096 &       0.1234 &       0.3678 &       0.2943 &        0.4043\\
                0.2246 &       0.3872 &       0.4015 &       0.1129 &        0.1550\\
                0.3645 &       0.1920 &       0.3781 &       0.0643 &        0.4240\\
                0.1784 &       0.4002 &       0.2786 &       0.3927 &       -0.2557\\
          \end{pmatrix}\rightarrow
          \begin{pmatrix}
                0.4096 &       0.1234 &       0.3678 &       0.2943 &        0.4043\\
                0.0000 &       0.3195 &       0.1998 &      -0.0485 &       -0.0667\\
                0.0000 &       0.0822 &       0.0508 &      -0.1976 &        0.0642\\
                0.0000 &       0.3465 &       0.1184 &       0.2645 &       -0.4318\\
          \end{pmatrix}\\
          \rightarrow\begin{pmatrix}
                0.4096 &       0.1234 &       0.3678 &       0.2943 &        0.4043\\
                0.0000 &       0.3465 &       0.1184 &       0.2645 &       -0.4318\\
                0.0000 &       0.0000 &       0.0227 &      -0.2603 &        0.1666\\
                0.0000 &       0.0000 &       0.0906 &      -0.2924 &        0.3315\\
          \end{pmatrix}\rightarrow
          \begin{pmatrix}
                0.4096 &       0.1234 &       0.3678 &       0.2943 &        0.4043\\
                0.0000 &       0.3465 &       0.1184 &       0.2645 &       -0.4318\\
                0.0000 &       0.0000 &       0.0906 &      -0.2924 &        0.3315\\
                0.0000 &       0.0000 &       0.0000 &      -0.1870 &        0.0835\\
          \end{pmatrix}
        \end{align*}
        回代解得
        \begin{align*}
            x_1&=     -0.1826 & x_2&=     -1.6632 & x_3&=      2.2178 & x_4&=     -0.4465
        \end{align*}

        实际上不限定计算小数精度的情况下的解为
        \begin{align*}
            x_1&=     -0.1819 & x_2&=     -1.6630 & x_3&=      2.2172 & x_4&=     -0.4467
        \end{align*}
        列主元消去法通过选择绝对值最大的主元避免了高斯消元法中大数除以小数的现象,从而提高了准确度。

        \end{itemize}
    \end{solution}
    \item[2.] \begin{itemize}
        \item[(1)] \begin{proof}
            由于矩阵 $\bm{A}$ 是对称矩阵,故令矩阵及其第一步约化矩阵为
            \begin{equation*}
                \bm{A}=
                \begin{pmatrix}
                    a_{11} & \bm{a}_{1}^\top \\
                    \bm{a}_1 & \bm{A}_1
                \end{pmatrix} \quad
                \bm{L}_1=
                \begin{pmatrix}
                    1 & \bm{0} \\
                    \bm{m}_1 & \bm{E} 
                \end{pmatrix}
            \end{equation*}
            则第一步约化后
            \begin{equation*}
                \bm{L}_1\bm{A}=\begin{pmatrix}
                    a_{11} & \bm{a}_1^\top \\
                    a_{11}\bm{m}_1+\bm{a}_1 & \bm{m}_1\bm{a}_1^\top+\bm{A}_1
                \end{pmatrix}=\begin{pmatrix}
                    a_{11} & \bm{a}_1^\top \\
                    \bm{0} & \bm{A}_2
                \end{pmatrix}
            \end{equation*}
            最后一个等号是高斯消元法的定义,则
            \begin{equation}\label{eq:gdef}
                a_{11}\bm{m}_1+\bm{a}_1=\bm{0} \quad \Rightarrow \quad  \bm{a}_1=-a_{11}\bm{m}_1 %\bm{m}_1=-\frac{1}{a_{11}}\bm{a}_1
            \end{equation}
            另一方面,
            \begin{equation*}
                \bm{A}_2=\bm{m}_1(-a_{11}\bm{m}_1)^\top+\bm{A}_1=-a_{11}\bm{m}_1\bm{m}_1^\top+\bm{A}_1%\bm{m}_1\bm{a}_1^\top+\bm{A}_1= -\frac{1}{a_{11}}\bm{a}_1\bm{a}_1^\top + \bm{A}_1
            \end{equation*}
            由于 $\bm{A}$ 是对称矩阵,则 $\bm{A}_1$ 也是对称矩阵,$\bm{A}_1=\bm{A}_1^\top$,考察
            \begin{equation*}
                \bm{A}_2^\top=-a_{11}(\bm{m}_1\bm{m}_1^\top)^\top+\bm{A}_1^\top=-a_{11}(\bm{m}_1^\top)^\top\bm{m}_1^\top+\bm{A}_1^\top=-a_{11}\bm{m}_1\bm{m}_1^\top+\bm{A}_1 % -\frac{1}{a_{11}}(\bm{a}_1\bm{a}_1^\top)^\top + \bm{A}_1^\top=-\frac{1}{a_{11}}(\bm{a}_1^\top)^\top\bm{a}_1^\top + \bm{A}_1^\top= -\frac{1}{a_{11}}\bm{a}_1\bm{a}_1^\top + \bm{A}_1
            \end{equation*}
            也就意味着
            \begin{equation*}
                \bm{A}_2=\bm{A}_2^\top
            \end{equation*}
            即 $\bm{A}_2$ 是对称矩阵。
        \end{proof}
        \item[(2)]\begin{solution}
            解方程组
            \begin{equation*}
                \begin{cases}
                    0.6428x_1+0.3475x_2-0.8468x_3=0.4127 \\
                    0.3475x_1+1.8423x_2+0.4759x_3=1.7321 \\
                    -0.8468x_1+0.4759x_2+1.2147x_3=-0.8621 
                \end{cases}
            \end{equation*}
            \begin{align*}
                \begin{pmatrix}
                    \bm{A} & \bm{b}
                \end{pmatrix}=&
                \begin{pmatrix}
                0.6428 &       0.3475 &      -0.8468 &        0.4127\\
                0.3475 &       1.8423 &       0.4759 &        1.7321\\
               -0.8468 &       0.4759 &       1.2147 &       -0.8621\\
          \end{pmatrix}\rightarrow
          \begin{pmatrix}
                0.6428 &       0.3475 &      -0.8468 &        0.4127\\
                0.0000 &       1.6544 &       0.9337 &        1.5090\\
                0.0000 &       0.9337 &       0.0992 &       -0.3184\\
          \end{pmatrix}\rightarrow\\
          &\begin{pmatrix}
                0.6428 &       0.3475 &      -0.8468 &        0.4127\\
                0.0000 &       1.6544 &       0.9337 &        1.5090\\
                0.0000 &       0.0000 &      -0.4278 &       -1.1700\\
          \end{pmatrix}
        \end{align*}
        回代解得
          \begin{align*}
          x_1&=      4.5867 & x_2&=     -0.6315 & x_3&=      2.7352
          \end{align*}
        \end{solution}
    \end{itemize}
    \item[6.] \begin{proof}
        由于 $\bm{A}$ 是对角占优阵,则它对角线上的元素
        \begin{equation}\label{eq:aii}
            |a_{ii}|>\sum_{\substack{j=1\\j\neq i}}^n |a_{ij}|
        \end{equation}
        式 \eqref{eq:aii} 在 $i=1$ 时的特殊情况
        \begin{equation}\label{eq:a11}
            |a_{11}|>\sum_{\substack{j=1\\j\neq 1}}^n |a_{1j}|=\sum_{j=2}^n |a_{1j}|
        \end{equation}
        由高斯消元法,对于 $\bm{A}$ 消去一步后的形式
        \begin{equation*}
            \begin{pmatrix}
                a_{11} & \bm{a}_1^\top \\
                \bm{0} & \bm{A}_2
            \end{pmatrix}
        \end{equation*}
        中的矩阵 $\bm{A}_2$ 元素有表达式
        \begin{equation}\label{eq:a22}
            a_{ij}^{(2)}=a_{ij}-\frac{a_{i1}}{a_{11}}a_{1j}
        \end{equation}
        式 \eqref{eq:a22} 在 $j=i$ 时的特殊情况
        \begin{equation}\label{eq:a22ii}
            a_{ii}^{(2)}=a_{ii}-\frac{a_{i1}}{a_{11}}a_{1i}
        \end{equation}
        考察矩阵 $\bm{A}_2$ 每一行初对角线上元素的绝对值的和
        \begin{align*}
            \sum_{\substack{j=2\\j\neq i}}^n \left|a_{ij}^{(2)}\right|&=\sum_{\substack{j=2\\j\neq i}}^n \left|a_{ij}-\frac{a_{i1}}{a_{11}}a_{1j}\right| & & \text{式 \eqref{eq:a22}}\\
            &=\sum_{\substack{j=2\\j\neq i}}^n \left(|a_{ij}|+\left|\frac{a_{i1}}{a_{11}}a_{1j}\right|\right) & &\text{三角不等式} |a\pm b|\leq |a|+|b|\\
            &=\sum_{\substack{j=2\\j\neq i}}^n |a_{ij}|+ \frac{|a_{i1}|}{|a_{11}|}\sum_{\substack{j=2\\j\neq i}}^n\left|a_{1j}\right|\\
            &<|a_{ii}|-|a_{i1}|+\frac{|a_{i1}|}{|a_{11}|}\sum_{\substack{j=2\\j\neq i}}^n \left|a_{1j}\right| && \text{式 \eqref{eq:aii}}\\
            &<|a_{ii}|-|a_{i1}|+\frac{|a_{i1}|}{|a_{11}|}(|a_{11}|-|a_{1i}|) && \text{式 \eqref{eq:a11}}\\
            &=|a_{ii}|-\frac{|a_{i1}|}{|a_{11}|}|a_{1i}|\\
            &\leq \left|a_{ii}-\frac{a_{i1}}{a_{11}}a_{1i}\right| && \text{不等式} |a|-|b|\leq \left||a|-|b|\right|\leq |a\pm b|\\
            &=\left|a_{ii}^{(2)}\right| && \text{式 \eqref{eq:a22ii}}
        \end{align*}
        这就证明了矩阵 $\bm{A}_2$ 是对角占优阵。
        
        结合习题 2 的结论和数学归纳法,对于对称的对角占优阵来说,用高斯消去法每一步产生的右下角矩阵都是对称的对角占优阵,对角占优性将保证小矩阵的第一个元素的绝对值是同行中最大的,结合对称性,将是同列中最大的,则其在列主元中也将被选为主元,从而得到与列主元消去法同样的计算过程。
    \end{proof}
    \item[7.] \begin{proof}
        \begin{itemize}
            \item[(1)] 由于 $\bm{A}$ 是对称正定矩阵,根据 Cholesky 分解,存在一个实的非奇异下三角矩阵 $\bm{L}=(l_{ij})_n$ 使得 $\bm{A}=\bm{L}\bm{L}^\top$,根据矩阵乘法运算规则,可以得到 $\bm{A}$ 对角线上的元素
            \begin{equation*}
                a_{ii}=\sum_{j=1}^i l_{ij}^2 > 0
            \end{equation*}
            其中不取等号是 $\bm{L}$ 为非奇异矩阵保证的,否则 $l_{ij}=0 (j=1,\cdots, n) \Rightarrow \det \bm{L}=0$ 不满足非奇异条件。

            \item[(2)] 考察高斯消元法第一步约化
            \begin{equation*}
                \bm{L}_1\bm{A}=\begin{pmatrix}
                    1 & \bm{0} \\
                    \bm{m}_1 & \bm{E} 
                \end{pmatrix}\bm{A}=\begin{pmatrix}
                    a_{11} & \bm{a}_1^\top \\
                    \bm{0} & \bm{A}_2
                \end{pmatrix}
            \end{equation*}
            根据习题 2(1) 的结论,$\bm{A}$ 是对称的,$\bm{A}_2$ 也是对称矩阵。
            
            由于 $\bm{A}$ 是正定的,则
            \begin{equation}\label{eq:xax}
                \forall \bm{x}\neq \bm{0}, \bm{x}^\top\bm{A}\bm{x}>0
            \end{equation}
            %由于 $\bm{m}_1$ 中的元素 $l_{i1}=-\frac{a_{i1}}{a_{11}}$
            由于矩阵 $\bm{L}_1$ 是非奇异的,所以 $\forall \bm{x}\neq \bm{0}, \bm{L}_1^\top\bm{x}\neq \bm{0}$(若 $\exists \bm{x}^*\neq \bm{0}, \bm{L}_1^\top\bm{x}^*=\bm{0}$,也就意味着 $\bm{L}$ 中的列向量是线性相关的,$\det \bm{L}=0$,违背了非奇异的定义),则根据 $\bm{A}$ 的正定性
            \begin{equation*}
                0<(\bm{L}_1^\top\bm{x})^\top \bm{A} \bm{L}_1^\top\bm{x}=\bm{x}^\top (\bm{L}_1 \bm{A}\bm{L}_1^\top)\bm{x}
            \end{equation*}
            则矩阵 $\bm{B}=\bm{L}_1\bm{A}\bm{L}_1^\top $ 也是正定的。展开
            \begin{equation*}
                \bm{B}=\begin{pmatrix}
                    a_{11} & \bm{a}_1^\top \\
                    \bm{0} & \bm{A}_2
                \end{pmatrix}\begin{pmatrix}
                    1 & \bm{m}_1^\top \\
                    \bm{0} & \bm{E} 
                \end{pmatrix}=\begin{pmatrix}
                    a_{11} & a_{11}\bm{m}_1^\top + \bm{a}_1^\top \\
                    \bm{0} & \bm{A}_2
                \end{pmatrix}=\begin{pmatrix}
                    a_{11} & \bm{0} \\
                    \bm{0} & \bm{A}_2
                \end{pmatrix}
            \end{equation*}
            最后一个等号由对称性导致的式 \eqref{eq:gdef} 得出。考察 $\bm{B}$ 的所有顺序主子式 $D_k=a_{11}D^\prime_{k-1}>0 (k=2,\cdots,n)$,由 (1) 可得 $a_{11}>0$,所以所有 $\bm{A}_2$ 对应的顺序主子式 $D^\prime_k>0$,故 $\bm{A}_2$ 是正定的。

            综上,$\bm{A}_2$ 是对称正定的。

            \item[(3)] 式 \eqref{eq:a22ii} 成立,由于 $\bm{A}$ 是对称矩阵,且 $a_{11}>0$,则
            \begin{equation*}
                a_{ii}^{(2)}=a_{ii}-\frac{a_{i1}^2}{a_{11}}\leq a_{ii}\quad (i=2,3,\cdots,n)
            \end{equation*}

            \item[(4)] 反证法。由于 $\bm{A}$ 是对称正定的,设 $a_{ij}=a_{ji}(i\neq j)$ 是 $\bm{A}$ 中绝对值最大的元素,在 (1) 中已经证明对角线上的元素大于0,$a_{ii}>0,a_{jj}>0$,有两种情形:
            \begin{enumerate}
                \item[i.] $a_{ij}=a_{ji}\geq\max(a_{ii},a_{jj})>0$,取式 \eqref{eq:xax} 中的 $\bm{x}$ 为
                \begin{equation*}
                    x_k=\begin{cases}
                        1, & k=i,\\
                        -1, & k=j, \\
                        0, & \text{其他}
                    \end{cases}
                \end{equation*}
                则
                \begin{equation*}
                    \bm{x}^\top\bm{A}\bm{x}=(a_{ii}-a_{ji})-(a_{ij}-a_{jj})=a_{ii}+a_{jj}-2a_{ij}\leq 0
                \end{equation*}
                \item[ii.] $a_{ij}=a_{ji}\leq-\max(a_{ii},a_{jj})$,取式 \eqref{eq:xax} 中的 $\bm{x}$ 为
                \begin{equation*}
                    x_k=\begin{cases}
                        1, & k=i,\\
                        1, & k=j, \\
                        0, & \text{其他}
                    \end{cases}
                \end{equation*}
                则
                \begin{equation*}
                    \bm{x}^\top\bm{A}\bm{x}=(a_{ii}+a_{ji})+(a_{ij}+a_{jj})=a_{ii}+a_{jj}+2a_{ij}\leq 0
                \end{equation*}
            \end{enumerate}
            不论如何都会与式 \eqref{eq:xax} 矛盾。所以 $\bm{A}$ 的绝对值最大的元素必在对角线上。

            \item[(5)] 第 (2) 小题已证明 $\bm{A}_2$ 是对称正定矩阵,结合第 (4) 小题对于对称正定矩阵,其绝对值最大的元素必在对角线上,令 $a_{pp}^{(2)} (p\in \{2,\cdots,n\})$ 是 $\bm{A}_2$ 中绝对值最大的元素,有
            \begin{equation}\label{eq:a2ij}
                \max_{2\leq i,j\leq n} \left|a_{ij}^{(2)}\right| = \left|a_{pp}^{(2)}\right|
            \end{equation}
            考虑第 (3) 小题的结果,以及第 (1) 小题的结果:对称正定矩阵对角线上的元素都是大于0的,
            \begin{equation}
                0<a_{pp}^{(2)}\leq a_{pp} \Rightarrow \left|a_{pp}^{(2)}\right|\leq\left|a_{pp}\right|
            \end{equation}
            考虑到 $p$ 的取值范围 $[2,\cdots,n]$,则根据最大值的定义,
            \begin{equation}\label{eq:aij}
                \left|a_{pp}\right|\leq \max_{2\leq i,j\leq n} |a_{ij}|
            \end{equation}
            综合式 \eqref{eq:a2ij} -- 式 \eqref{eq:aij},有
            \begin{equation*}
                \max_{2\leq i,j\leq n} \left|a_{ij}^{(2)}\right|\leq \max_{2\leq i,j\leq n} |a_{ij}|
            \end{equation*}

            \item[(6)](目测题目有误,应该是 $\forall i,j\quad |a_{ij}|<1$,即 $\max_{1\leq i,j\leq n} |a_{ij}|<1$)%保持跟第 (5) 小题相同的符号意义,
            % TODO

            当 $i=1$ 时,$\left|a_{1j}^{(2)}\right|=|a_{1j}|<1$;当 $2\leq i\leq n, j=1$ 时,$\left|a_{i1}^{(2)}\right|=0<1$;
            当 $2\leq i,j\leq n$ 时,
            \begin{equation*}
                \left|a_{ij}^{(2)}\right|\leq \max_{2\leq i,j\leq n} \left|a_{ij}^{(2)}\right|
                %=\left|a_{pp}^{(2)}\right|\leq |a_{pp}|
                \leq \max_{2\leq i,j\leq n} |a_{ij}|\leq \max_{1\leq i,j\leq n} |a_{ij}|<1
            \end{equation*}
            类似的,根据数学归纳法,将得到 $\forall k, \left|a_{ij}^{(k)}\right|<1$。
        \end{itemize}
    \end{proof}

    \item[9.] \begin{proof}
        设 $\bm{A}$ 的 Crout 分解为
        \begin{equation*}
            \bm{A}=\begin{pmatrix}
                l_{11} \\
                l_{21} & l_{22} \\
                \vdots & \ddots & \ddots \\
                l_{n,1} & \cdots & l_{n,n-1} & l_{nn}
            \end{pmatrix}
            \begin{pmatrix}
                1 & \cdots & u_{1,n-1} & u_{1n} \\
                  & 1 & \cdots & u_{2n} \\
                  &   &  \ddots & \vdots \\
                  &   &         & 1
            \end{pmatrix}
        \end{equation*}
        使用数学归纳法。
        \paragraph{起步} 可知
        \begin{equation*}
            a_{i1} = l_{i1}\quad (i=1,2,\cdots,n)
        \end{equation*}
        则
        \begin{equation*}
            a_{1j} = l_{11}u_{1j} \Rightarrow u_{1j}= a_{1j} / l_{11} \quad (j=2,\cdots,n)
        \end{equation*}
        从而得到 $\bm{L}$ 的第 1 行列和 $\bm{U}$ 的第 1 行。
        \paragraph{迭代} 设已经定出 $\bm{L}$ 的前 $r-1$ 列和 $\bm{U}$ 的前 $r-1$ 行,则
        \begin{equation*}
            a_{ir} = \sum_{k=1}^n l_{ik}u_{kr} = \sum_{k=1}^{r-1}l_{ik}u_{kr}+l_{ir} \Rightarrow l_{ir}=a_{ir}-\sum_{k=1}^{r-1}l_{ik}u_{kr}\quad (i=r,r+1,\cdots,n)
        \end{equation*}
        从而得到 $\bm{L}$ 的第 $r$ 列;另一方面,
        \begin{equation*}
            a_{rj}= \sum_{k=1}^n l_{rk}u_{kj}=\sum_{k=1}^{r-1} l_{rk}u_{kj}+l_{rr}u_{rj} \Rightarrow u_{rj}=\left(a_{ri}-\sum_{k=1}^{r-1} l_{rk}u_{kj}\right)/l_{rr}\quad (j=r+1,\cdots,n)
        \end{equation*}
        从而得到 $\bm{U}$ 的第 $r$ 行。
        
        \paragraph{结论} 根据数学归纳法原理,可以得到 $\bm{A}$ 的 Crout 分解。
    \end{proof}

    \item[10.] \begin{solution}
        \begin{itemize}
            \item[(1)] 若 $\bm{U}$ 为上三角矩阵,则
            \begin{equation*}
                \begin{pmatrix}
                    u_{11} & \cdots & u_{1,n-1} & u_{1n} \\
                      & u_{22} & \cdots & u_{2n} \\
                      &   &  \ddots & \vdots \\
                      &   &         & u_{nn}
                \end{pmatrix}
                \begin{pmatrix}
                    x_1\\x_2\\ \vdots \\ x_n
                \end{pmatrix}
                =
                \begin{pmatrix}
                    d_1\\d_2\\ \vdots \\ d_n
                \end{pmatrix}
            \end{equation*}
            求解方法为从下而上,先计算 $x_n=d_n/u_{nn}$,再代入上一行,$u_{n-1,n-1}x_{n-1}+u_{n-1,n}x_{n}=d_{n-1}\Rightarrow x_{n-1}=\frac{d_{n-1}-u_{n-1,n-1}x_{n-1}}{u_{n-1,n-1}}$;$\cdots$ 类似地,有
            \begin{equation*}
                x_{i}=\frac{d_{i}-\sum_{j=i+1}^{n}u_{ij}x_j}{u_{ii}}\quad (i=n-1,n-2,\cdots,1)
            \end{equation*}
            若 $\bm{U}$ 为下三角矩阵,则为从上到下,$x_1=d_1/u_{11}$,
            \begin{equation*}
                x_{i}=\frac{d_i-\sum_{j=1}^{i-1}u_{ij}x_j}{u_{ii}}\quad (i=2,3,\cdots,n)
            \end{equation*}

            \item[(2)] 乘法:$0+1+\cdots+n-1$,除法:$n$,乘除法总计:$\frac{n^2+n}{2}$。
            \item[(3)] 若 $\bm{U}$ 为上三角矩阵,它的逆矩阵 $\bm{S}=\bm{U}^{-1}$ 也是上三角矩阵 % 为什么
            ,设
            \begin{equation*}
                \begin{pmatrix}
                    u_{11} & \cdots & u_{1,n-1} & u_{1n} \\
                      & u_{22} & \cdots & u_{2n} \\
                      &   &  \ddots & \vdots \\
                      &   &         & u_{nn}
                \end{pmatrix}
                \begin{pmatrix}
                    s_{11} & \cdots & s_{1,n-1} & s_{1n} \\
                      & s_{22} & \cdots & s_{2n} \\
                      &   &  \ddots & \vdots \\
                      &   &         & s_{nn}
                \end{pmatrix}=\bm{E}
            \end{equation*}
            则
            \begin{align*}
                u_{ii}s_{ii}=1 \Rightarrow s_{ii}&=\frac{1}{u_{ii}} & (i=1,2,\cdots,n) \\
                s_{ij}&=-\frac{\sum_{k=i+1}^n u_{ik}s_{kj}}{u_{ii}} & (i=n-1,n-2,\cdots,1;j=i+1,i+2,\cdots,n)
            \end{align*}
            若 $\bm{U}$ 为下三角矩阵,同理可得
            \begin{align*}
                u_{ii}s_{ii}=1 \Rightarrow s_{ii}&=\frac{1}{u_{ii}} & (i=1,2,\cdots,n) \\
                s_{ij}&=-\frac{\sum_{k=1}^{i-1} u_{ik}s_{kj}}{u_{ii}} & (i=2,3,\cdots,n;j=1,2,\cdots,i-1)
            \end{align*}
        \end{itemize}
    \end{solution}
    \item[11.] \begin{proof}
        \begin{itemize}
            \item[(1)] 
            由于 $\bm{A}=\bm{A}^\top$,则
            \begin{equation*}
                (\bm{A}^{-1})^\top = (\bm{A}^\top)^{-1} = \bm{A}^{-1}
            \end{equation*}
            所以 $\bm{A}^{-1}$ 是对称的。

            若 $\bm{A}$ 是对称正定矩阵,则存在非奇异的下三角矩阵 $\bm{L}$ 满足它的 Cholesky 分解:
            \begin{equation*}
                \bm{A}=\bm{L}\bm{L}^\top
            \end{equation*}
            考虑它的逆,由于 $\bm{L}$ 非奇异所以它存在逆矩阵 $\bm{L}^{-1}$,
            \begin{equation*}
                \bm{A}^{-1}=(\bm{L}^\top)^{-1}\bm{L}^{-1}=(\bm{L}^{-1})^\top\bm{L}^{-1}
            \end{equation*}
            % 考察它的转置,
            % \begin{equation*}
            %     (\bm{A}^{-1})^\top = ((\bm{L}^{-1})^\top\bm{L}^{-1})^\top = (\bm{L}^{-1})^\top\bm{L}^{-1}
            % \end{equation*}
            % 故 $(\bm{A}^{-1})^\top=\bm{A}^{-1}$,即 $\bm{A}^{-1}$ 是对称的。
            考察 $\forall x\neq 0$,
            \begin{equation*}
                \bm{x}^\top \bm{A}^{-1}\bm{x}=\bm{x}^\top (\bm{L}^{-1})^\top\bm{L}^{-1} \bm{x}=(\bm{L}^{-1} \bm{x})^\top (\bm{L}^{-1} \bm{x})>0
            \end{equation*}
            易知 $\bm{L}^{-1}$ 也是非奇异的,所以 $\bm{L}^{-1}\bm{x}\neq 0$,最后一个大于号成立。故 $\bm{A}^{-1}$ 是正定的。

            

            % 由于 $\bm{A}$ 是正定矩阵,所以它的特征值 $\lambda_i>0$,则 $\bm{A}^{-1}$ 的特征值 $\frac{1}{\lambda_i}>0$,那么 $\bm{A}^{-1}$ 也是正定矩阵。

            综上,$\bm{A}^{-1}$ 是对称正定的。
            \item[(2)] 由于 $\bm{A}$ 是对称正定的,所以 $\bm{A}^{-1}$ 也是对称正定的(上一小问的结论),存在它的 Cholesky 分解,使得 $L^*$ 对角线上的元素都是正的且唯一:
            \begin{equation*}
                \bm{A}^{-1}=\bm{L}^*{\bm{L}^*}^\top
            \end{equation*}
            则
            \begin{equation*}
                \bm{A}=(\bm{A}^{-1})^{-1}=(\bm{L}^*{\bm{L}^*}^\top)^{-1}=({\bm{L}^*}^\top)^{-1}{\bm{L}^*}^{-1}=({\bm{L}^*}^{-1})^\top{\bm{L}^*}^{-1}
            \end{equation*}
            其中 ${\bm{L}^*}^{-1}$ 的对角线元素均正(根据矩阵乘法,$a_{ii}a_{ii}^*=1$,$a_{ii}>0, a_{ii}^*=\frac{1}{a_{ii}}>0$)。
        \end{itemize}
    \end{proof}
    \item[13.] \begin{solution}
        \begin{equation*}
            \bm{A}=\begin{pmatrix}
                2 & -1 \\
                -1 & 2 & -1 \\
                & -1 & 2 & -1 \\
                &  & -1 & 2 & -1 \\
                &  &   &  -1 & 2
            \end{pmatrix}=\begin{pmatrix}
                2 \\
                -1 & \frac{3}{2} \\
                   & -1 & \frac{4}{3} \\
                   &    &    -1 & \frac{5}{4} \\
                   &    &             & -1 & \frac{6}{5}
            \end{pmatrix}\begin{pmatrix}
                1 & -\frac{1}{2} \\
                  &  1 & -\frac{2}{3} \\
                  &    &  1  & -\frac{3}{4} \\
                  &    &     &   1   & -\frac{4}{5} \\
                  &    &     &       &    1
            \end{pmatrix}
        \end{equation*}
        解 $\bm{L}\bm{y}=\bm{f}$:
        \begin{equation*}
            \begin{pmatrix}
                2 \\
                -1 & \frac{3}{2} \\
                   & -1 & \frac{4}{3} \\
                   &    &    -1 & \frac{5}{4} \\
                   &    &             & -1 & \frac{6}{5}
            \end{pmatrix}\begin{pmatrix}
                y_1 \\ y_2 \\ y_3 \\ y_4 \\ y_5
            \end{pmatrix}=\begin{pmatrix}
                1 \\ 0 \\ 0 \\ 0 \\ 0
            \end{pmatrix}\Rightarrow\begin{pmatrix}
                y_1 \\ y_2 \\ y_3 \\ y_4 \\ y_5
            \end{pmatrix}=\begin{pmatrix}
                \frac{1}{2} \\ \frac{1}{3} \\ \frac{1}{4} \\ \frac{1}{5} \\ \frac{1}{6}
            \end{pmatrix}
        \end{equation*}
        解 $\bm{U}\bm{x}=\bm{y}$:
        \begin{equation*}
            \begin{pmatrix}
                1 & -\frac{1}{2} \\
                  &  1 & -\frac{2}{3} \\
                  &    &  1  & -\frac{3}{4} \\
                  &    &     &   1   & -\frac{4}{5} \\
                  &    &     &       &    1
            \end{pmatrix}\begin{pmatrix}
                x_1 \\ x_2 \\ x_3 \\ x_4 \\ x_5
            \end{pmatrix}=\begin{pmatrix}
                \frac{1}{2} \\ \frac{1}{3} \\ \frac{1}{4} \\ \frac{1}{5} \\ \frac{1}{6}
            \end{pmatrix}\Rightarrow\begin{pmatrix}
                x_1 \\ x_2 \\ x_3 \\ x_4 \\ x_5
            \end{pmatrix}=\begin{pmatrix}
                \frac{5}{6} \\ \frac{2}{3} \\ \frac{1}{2} \\ \frac{1}{3} \\ \frac{1}{6}
            \end{pmatrix}
        \end{equation*}
    \end{solution}
    \item[14.] \begin{solution}
        \begin{equation*}
            \bm{A}=\begin{pmatrix}
                2 & -1 & 1 \\
                -1 & -2 & 3 \\
                1 & 3 & 1
            \end{pmatrix}=\begin{pmatrix}
                1 \\
                -\frac{1}{2} & 1 \\
                \frac{1}{2} & -\frac{7}{5} & 1
            \end{pmatrix}
            \begin{pmatrix}
                2 \\
                & -\frac{5}{2} \\
                & & \frac{27}{5}
            \end{pmatrix}
            \begin{pmatrix}
                1 & -\frac{1}{2} & \frac{1}{2}\\
                & 1 & -\frac{7}{5}\\
                & & 1
            \end{pmatrix}=\bm{L}\bm{D}\bm{L}^\top
        \end{equation*}
        求解 $\bm{L}\bm{y}=\bm{b}$,
        \begin{equation*}
            \begin{pmatrix}
                1 \\
                -\frac{1}{2} & 1 \\
                \frac{1}{2} & -\frac{7}{5} & 1
            \end{pmatrix}
            \begin{pmatrix}
                y_1 \\
                y_2 \\
                y_3
            \end{pmatrix}=
            \begin{pmatrix}
                4 \\
                5 \\
                6
            \end{pmatrix}\Rightarrow
            \begin{pmatrix}
                y_1 \\
                y_2 \\
                y_3
            \end{pmatrix}=\begin{pmatrix}
                4 \\
                7 \\
                \frac{69}{5}
            \end{pmatrix}
        \end{equation*}
        求解 $\bm{DL}^\top\bm{x}=\bm{y}$,
        \begin{equation*}
            \begin{pmatrix}
                2 \\
                & -\frac{5}{2} \\
                & & \frac{27}{5}
            \end{pmatrix}
            \begin{pmatrix}
                1 & -\frac{1}{2} & \frac{1}{2}\\
                & 1 & -\frac{7}{5}\\
                & & 1
            \end{pmatrix}\begin{pmatrix}
                x_1 \\
                x_2 \\
                x_3
            \end{pmatrix}=\begin{pmatrix}
                4 \\
                7 \\
                \frac{69}{5}
            \end{pmatrix}\Rightarrow
            \begin{pmatrix}
                x_1 \\
                x_2 \\
                x_3
            \end{pmatrix}=\begin{pmatrix}
                \frac{10}{9} \\
                \frac{7}{9} \\
                \frac{23}{9}
            \end{pmatrix}
        \end{equation*}
    \end{solution}
    \item[15.]\begin{solution}
        \begin{equation*}
            \bm{A}=\begin{pmatrix}
                1 & 2 & 3 \\
                2 & 4 & 1 \\
                4 & 6 & 7
            \end{pmatrix}
        \end{equation*}
        的 $D_1\neq 0, D_2=0, D_3\neq 0$,所以 $\bm{A}$ 不能进行 Doolittle 分解。
        \begin{equation*}
            \bm{B}=\begin{pmatrix}
                1 & 1 & 1 \\
                2 & 2 & 1 \\
                3 & 3 & 1 \\
            \end{pmatrix}
        \end{equation*}
        的 $D_1\neq 0,D_2 = 0,D_3=0$,存在 Doolittle 分解但不唯一, 
        \begin{equation*}
            \bm{B}=\begin{pmatrix}
                1 \\
                2 & 1 \\
                3 & b & 1
            \end{pmatrix}
            \begin{pmatrix}
                1 & 1 & 1 \\
                  &  & -1 \\
                  &  & b-2
            \end{pmatrix}
        \end{equation*}
        其中 $b$ 为任意实数。
        \begin{equation*}
            \bm{C}=\begin{pmatrix}
                1 & 2 & 6 \\
                2 & 5 & 15 \\
                6 & 15 & 46
            \end{pmatrix}
        \end{equation*}
        的 $D_1=1, D_2=1, D_3=1$,故 $\bm{C}$ 有唯一的 Doolittle 分解:
        \begin{equation*}
            \bm{C}=\begin{pmatrix}
                1 \\
                2 & 1 \\
                6 & 3 & 1
            \end{pmatrix}\begin{pmatrix}
                1 & 2 & 6 \\
                   & 1 & 3 \\
                  &  & 1 
            \end{pmatrix}
        \end{equation*}
    \end{solution}
    \item[18.] \begin{solution}
        \begin{equation*}
            \bm{A}=\begin{pmatrix}
                0.6 & 0.5 \\
                0.1 & 0.3
            \end{pmatrix}
        \end{equation*}
        行范数
        \begin{equation*}
            ||\bm{A}||_{\infty}=1.1
        \end{equation*}
        列范数
        \begin{equation*}
            ||\bm{A}||_1=0.8
        \end{equation*}
        2-范数:由于
        \begin{align*}
            \bm{A}^\top\bm{A}&=\begin{pmatrix}
                0.37 & 0.33 \\
                0.33 & 0.34
            \end{pmatrix}\\
            \det(\lambda\bm{E}-\bm{A}^\top\bm{A})&=0\Rightarrow \lambda_{1} = 0.68534073, \lambda_2 = 0.02465927
        \end{align*}
        \begin{equation*}
            ||\bm{A}||_2=\sqrt{0.68534073}=0.82785
        \end{equation*}
        $F$-范数
        \begin{equation*}
            ||\bm{A}||_F=\sqrt{0.6^2+0.5^2+0.1^2+0.3^2}=0.842615
        \end{equation*}
    \end{solution}
    \item[20.]\begin{proof}
        \begin{description}
            \item[正定条件] $\Vert\bm{x}\Vert_{\bm{P}}=\lVert\bm{P}\bm{x}\rVert\geq 0$ 显然。因为 $\det\bm{P}\neq0$,所以$\Vert\bm{x}\Vert_{\bm{P}}=0\Leftrightarrow \bm{P}\bm{x}=0\Leftrightarrow \bm{x}=0$。
            \item[齐次条件] $\Vert \alpha\bm{x}\Vert_{\bm{P}}=\Vert \bm{P}\alpha\bm{x}\Vert=\Vert \alpha\bm{P}\bm{x}\Vert=\alpha\Vert \bm{P}\bm{x}\Vert=\alpha\Vert\bm{x}\Vert_{\bm{P}},\forall \alpha\in \mathbf{R}$
            \item[三角不等式]  $\Vert \bm{x}+\bm{y}\Vert_{\bm{P}}=\Vert\bm{P}(\bm{x}+\bm{y})\Vert=\Vert\bm{P}\bm{x}+\bm{P}\bm{y}\Vert\leq \Vert\bm{P}\bm{x}\Vert+\Vert\bm{P}\bm{y}\Vert=\Vert\bm{x}\Vert_{\bm{P}}+\Vert\bm{y}\Vert_{\bm{P}}$
        \end{description}
        故 $\Vert\bm{x}\Vert_{\bm{P}}$ 是 $\mathbf{R}^n$ 上向量的一种范数。
    \end{proof}
    \item[23.]\begin{solution}
        令 $\bm{x}=(x,y)$,
        \begin{align*}
            \Vert \bm{x}\Vert_1 &= 1 &\Rightarrow |x|+|y|&=1\\
            \Vert \bm{x}\Vert_2 &= 1 &\Rightarrow \sqrt{x^2+y^2}&=1\\
            \Vert \bm{x}\Vert_\infty &= 1 &\Rightarrow \max\{x,y\}&=1
        \end{align*}
        \begin{figure}[H]
            \begin{subfigure}{.33\textwidth}
                \begin{tikzpicture}
                    \begin{axis}[axis x line={middle},
                    axis y line={middle},
                    width={6cm},
                    height={6cm},
                    xmin={-2},
                    xmax={2},
                    ymin={-2},
                    ymax={2},
                    xlabel={$x$},
                    ylabel={$y$},
                    ]
                     \addplot [domain=-1:1,thick,] {1-abs(x)};
                     \addplot [domain=-1:1,thick,] {abs(x)-1};
                    \end{axis}
                \end{tikzpicture}
                \caption{$\Vert \bm{x}\Vert_1 = 1 $}
            \end{subfigure}
            \begin{subfigure}{.33\textwidth}
                \begin{tikzpicture}
                    \begin{axis}[axis x line={middle},
                    axis y line={middle},
                    width={6cm},
                    height={6cm},
                    xmin={-2},
                    xmax={2},
                    ymin={-2},
                    ymax={2},
                    xlabel={$x$},
                    ylabel={$y$},
                    ]
                     \addplot [domain=-1:1,thick,samples=201,] {sqrt(1-x^2)};
                     \addplot [domain=-1:1,thick,samples=201,] {-sqrt(1-x^2)};
                    \end{axis}
                    \end{tikzpicture}     
                \caption{$\Vert \bm{x}\Vert_2 = 1 $}               
            \end{subfigure}
            \begin{subfigure}{.33\textwidth}
                \begin{tikzpicture}
                    \begin{axis}[axis x line={middle},
                    axis y line={middle},
                    width={6cm},
                    height={6cm},
                    xmin={-2},
                    xmax={2},
                    ymin={-2},
                    ymax={2},
                    xlabel={$x$},
                    ylabel={$y$},
                    ]
                    \draw[thick] (-1,1) rectangle (1,-1);
                    \end{axis}
                \end{tikzpicture}
                \caption{$\Vert \bm{x}\Vert_\infty = 1 $}
            \end{subfigure}
        \end{figure}
    \end{solution}
    \item[29.] \begin{proof}
        \begin{align*}
            \bm{A}&=\begin{pmatrix}
                2\lambda & \lambda \\
                1 & 1
            \end{pmatrix}&
            \quad \bm{A}^{-1}&=\begin{pmatrix}
                \frac{1}{\lambda} & -1 \\
                -\frac{1}{\lambda} & 2
            \end{pmatrix}\\
            \Vert\bm{A}\Vert_\infty&=\max\{|3\lambda|, 2\} &\Vert\bm{A}^{-1}\Vert_\infty&=\max\left\{\left|\frac{1}{\lambda}\right|+1,\left|-\frac{1}{\lambda}\right|+2\right\}=\left|\frac{1}{\lambda}\right|+2
        \end{align*}
        % \begin{figure}[H]
        %     \centering
        %     \begin{tikzpicture}
        %         \begin{axis}[ymin={-1},
        %         xtick={-0.667,0,0.667},
        %         xticklabels={$-\frac23$,$0$,$\frac23$},
        %         axis x line={middle},
        %         axis y line={middle},
        %         ymax={9},
        %         xlabel={$\lambda$},
        %         legend pos={outer north east},
        %         no markers,]
        %          \addplot+ [thick,] coordinates { (-1,3) (-0.667,2) (0.667,2) (1,3)};
        %         %  \addplot [red,thick,domain=0.1:0.667,] {1/x-1};
        %         %  \addplot [red,thick,domain=0.667:1.5,] {2-1/x};
        %         %  \addplot [red,thick,domain=-0.667:-0.1,] {-1/x-1};
        %         %  \addplot [red,thick,domain=-1.5:-0.667,] {2+1/x};
        %         \addplot [red,thick,domain=0.1:1,] {1/x+2};
        %         \addplot [red,thick,domain=-1:-0.1,] {-1/x+2};
        %          \legend{$\Vert \bm{A}\Vert_\infty$,$\Vert \bm{A}^{-1}\Vert_\infty$,}
        %         \end{axis}
        %     \end{tikzpicture}
        % \end{figure}
        \begin{equation*}
            \text{cond}(\bm{A})_\infty=\Vert \bm{A}^{-1}\Vert_\infty\Vert \bm{A}\Vert_\infty=\left(\left|\frac{1}{\lambda}\right|+2\right)\cdot \max\{|3\lambda|, 2\}=\begin{cases}
                \left(\left|\frac{1}{\lambda}\right|+2\right)\cdot |3\lambda|=3+6|\lambda|,& |\lambda|\geq \frac{2}{3},\\
                2\left(\left|\frac{1}{\lambda}\right|+2\right),& |\lambda|<\frac{2}{3}
            \end{cases}
        \end{equation*}
        三个分段分别单调,均在边界处取得最小值,即在 $\lambda=\pm \frac{2}{3}$ 处取得最小值 $7$。
    \end{proof}
    \item[31.]\begin{solution}
        \begin{align*}
            \bm{A}&=\begin{pmatrix}
                100 & 99 \\
                99 & 98
            \end{pmatrix}\\
            %&\bm{A}^{-1}&=\begin{pmatrix}
            %    -98 & 99 \\
            %    99 & -100
            %\end{pmatrix}\\
            \bm{A}^\top\bm{A}=\bm{A}^2&=\begin{pmatrix}
                19801 & 19602 \\
                19602 & 19405
            \end{pmatrix}\\
            %&(\bm{A}^{-1})^\top\bm{A}^{-1}=(\bm{A}^{-1})^2&=\begin{pmatrix}
            %    19405 & -19602 \\
            %    -19602 &  19801
            %\end{pmatrix}\\
            \det(\lambda E-\bm{A}^\top\bm{A})&=0\\
            %&\det(\lambda E-(\bm{A}^{-1})^\top\bm{A}^{-1})&=0\\
            \lambda_1&=3.92\times 10^4 & \\
            \lambda_2&=2.55\times 10^{-5} &
        \end{align*}
        根据 $\bm{A}$ 是对称矩阵,$\lambda_1\lambda_2=\det(\bm{A}^\top\bm{A})=\det(\bm{A}^2)=\det(\bm{A})^2=(-1)^2=1$,故谱条件数:
        \begin{equation}
            \text{cond}(\bm{A})_2=\sqrt{\frac{\lambda_1}{\lambda_2}}=\sqrt{\lambda_1^2}=\lambda_1=3.92\times 10^4 
        \end{equation}
        \begin{align*}
            \bm{A}&=\begin{pmatrix}
                100 & 99 \\
                99 & 98
            \end{pmatrix}
            &\bm{A}^{-1}&=\begin{pmatrix}
                -98 & 99 \\
                99 & -100
            \end{pmatrix}\\
            \Vert\bm{A}\Vert_\infty&=199&\Vert\bm{A}^{-1}\Vert_\infty&=199
        \end{align*}
        则
        \begin{equation}
            \text{cond}(\bm{A})_\infty=\Vert\bm{A}^{-1}\Vert_\infty\Vert\bm{A}\Vert_\infty=39601
        \end{equation}
    \end{solution}
    \item[32.]\begin{proof}
        由于 $\bm{A}$ 是正交矩阵,则
        \begin{equation*}
            \bm{A}^\top\bm{A}=\bm{E}
        \end{equation*}
        则谱条件数
        \begin{equation*}
            \text{cond}(\bm{A})_2=\sqrt{\frac{\lambda_{\text{max}}(\bm{A}^\top\bm{A})}{\lambda_{\text{min}}(\bm{A}^\top\bm{A})}}=\sqrt{\frac{\lambda_{\text{max}}(\bm{E})}{\lambda_{\text{min}}(\bm{E})}}=\sqrt{\frac{1}{1}}=1
        \end{equation*}
    \end{proof}
\end{itemize}

\end{document}