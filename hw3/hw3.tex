\documentclass{sjtuarticle}
\usepackage{ntheorem}
\usepackage{float}
\usepackage{bm}
\title{作业3}
\author{李子龙\\123033910195}
\begin{document}
\maketitle
% P199: 1,2,6,7,9,10,11,13,14,15,18,20,23,29,31,32

\begin{itemize}
    \item[1.] 
% 0.4096x_1+0.1234x_2+0.3678x_3+0.2943x_4=0.4043
% 0.2246x_1+0.3872x_2+0.4015x_3+0.1129x_4=0.1550
% 0.3645x_1+0.1920x_2+0.3781x_3+0.0643x_4=0.4240
% 0.1784x_1+0.4002x_2+0.2786x_3+0.3927x_4=-0.2557
    \begin{solution}
        解方程组
        \begin{equation*}
            \begin{cases}
                0.4096x_1+0.1234x_2+0.3678x_3+0.2943x_4=0.4043 \\
                0.2246x_1+0.3872x_2+0.4015x_3+0.1129x_4=0.1550 \\
                0.3645x_1+0.1920x_2+0.3781x_3+0.0643x_4=0.4240 \\
                0.1784x_1+0.4002x_2+0.2786x_3+0.3927x_4=-0.2557
            \end{cases}
        \end{equation*}
        \begin{itemize}
            \item[(1)] 高斯消元法:
        \begin{align*}
            \begin{pmatrix}
                \bm{A} & \bm{b}
            \end{pmatrix}=\begin{pmatrix}
                0.4096 &       0.1234 &       0.3678 &       0.2943 &        0.4043\\
                0.2246 &       0.3872 &       0.4015 &       0.1129 &        0.1550\\
                0.3645 &       0.1920 &       0.3781 &       0.0643 &        0.4240\\
                0.1784 &       0.4002 &       0.2786 &       0.3927 &       -0.2557\\
            \end{pmatrix}\rightarrow
            \begin{pmatrix}
                0.4096 &       0.1234 &       0.3678 &       0.2943 &        0.4043\\
                0.0000 &       0.3195 &       0.1998 &      -0.0485 &       -0.0667\\
                0.0000 &       0.0822 &       0.0508 &      -0.1976 &        0.0642\\
                0.0000 &       0.3465 &       0.1184 &       0.2645 &       -0.4318\\
            \end{pmatrix} \\
            \rightarrow\begin{pmatrix}
            0.4096 &       0.1234 &       0.3678 &       0.2943 &        0.4043\\
            0.0000 &       0.3195 &       0.1998 &      -0.0485 &       -0.0667\\
            0.0000 &       0.0000 &      -0.0006 &      -0.1851 &        0.0814\\
            0.0000 &       0.0000 &      -0.0983 &       0.3171 &       -0.3595\\
            \end{pmatrix}\rightarrow
            \begin{pmatrix}
            0.4096 &       0.1234 &       0.3678 &       0.2943 &        0.4043\\
            0.0000 &       0.3195 &       0.1998 &      -0.0485 &       -0.0667\\
            0.0000 &       0.0000 &      -0.0006 &      -0.1851 &        0.0814\\
            0.0000 &       0.0000 &       0.0000 &      30.6427 &      -13.6955\\
            \end{pmatrix}
        \end{align*}

        \begin{align*}
            x_1&=     -0.1800 & x_2&=     -1.6617 & x_3&=      2.2148 & x_4&=     -0.4469
        \end{align*}
        \item[(2)] 列主元消元法:
        \begin{align*}
            \begin{pmatrix}
                \bm{A} & \bm{b}
            \end{pmatrix}=
            \begin{pmatrix}
                0.4096 &       0.1234 &       0.3678 &       0.2943 &        0.4043\\
                0.2246 &       0.3872 &       0.4015 &       0.1129 &        0.1550\\
                0.3645 &       0.1920 &       0.3781 &       0.0643 &        0.4240\\
                0.1784 &       0.4002 &       0.2786 &       0.3927 &       -0.2557\\
          \end{pmatrix}\rightarrow
          \begin{pmatrix}
                0.4096 &       0.1234 &       0.3678 &       0.2943 &        0.4043\\
                0.0000 &       0.3195 &       0.1998 &      -0.0485 &       -0.0667\\
                0.0000 &       0.0822 &       0.0508 &      -0.1976 &        0.0642\\
                0.0000 &       0.3465 &       0.1184 &       0.2645 &       -0.4318\\
          \end{pmatrix}\\
          \rightarrow\begin{pmatrix}
                0.4096 &       0.1234 &       0.3678 &       0.2943 &        0.4043\\
                0.0000 &       0.3465 &       0.1184 &       0.2645 &       -0.4318\\
                0.0000 &       0.0000 &       0.0227 &      -0.2603 &        0.1666\\
                0.0000 &       0.0000 &       0.0906 &      -0.2924 &        0.3315\\
          \end{pmatrix}\rightarrow
          \begin{pmatrix}
                0.4096 &       0.1234 &       0.3678 &       0.2943 &        0.4043\\
                0.0000 &       0.3465 &       0.1184 &       0.2645 &       -0.4318\\
                0.0000 &       0.0000 &       0.0906 &      -0.2924 &        0.3315\\
                0.0000 &       0.0000 &       0.0000 &      -0.1870 &        0.0835\\
          \end{pmatrix}
        \end{align*}

        \begin{align*}
            x_1&=     -0.1826 & x_2&=     -1.6632 & x_3&=      2.2178 & x_4&=     -0.4465
        \end{align*}

        实际上不限定计算小数精度的情况下的解为
        \begin{align*}
            x_1&=     -0.1819 & x_2&=     -1.6630 & x_3&=      2.2172 & x_4&=     -0.4467
        \end{align*}
        列主元消去法通过选择绝对值最大的主元避免了高斯消元法中大数除以小数的现象,从而提高了准确度。

        \end{itemize}
    \end{solution}
    \item[2.] \begin{itemize}
        \item[(1)] \begin{proof}
            
        \end{proof}
    \end{itemize}
\end{itemize}

\end{document}