\documentclass{sjtuarticle}
\allowdisplaybreaks[1]
\usepackage{array}
\usepackage{ntheorem}
\usepackage{float}
\usepackage{pgfplots}
\pgfplotsset{compat=newest}
\usepackage{bm}
\usepackage{booktabs}
\usepackage{subcaption}
\usepackage[colorlinks]{hyperref}

\def\dd{\mathrm{d}}
\def\DD{\mathrm{D}}
\def\ee{\mathrm{e}}

\title{作业9}
\author{Log Creative}
\date{2024 年 1 月 4 日}
\begin{document}
\maketitle

% 作业:P246: 1(1),3,9(1)修正见附件,10

\begin{itemize}
    \item[1(1).] \begin{solution}
        \begin{equation*}
        \mathbf{A}_1=\begin{pmatrix}
            7 & 3 & -2 \\ 3 & 4 & -1 \\ -2 & -1 & 3
        \end{pmatrix}
    \end{equation*}
    计算如表所示:
    \begin{table}[h]
        \centering
        \begin{tabular}{ccc}
            \toprule
            $k$ & $\mathbf{u}_k^\top$ & $\max(\mathbf{v}_k)$ \\
            \midrule
            0 & $(1,1,1)$ & \\
1     &   $(1  , 0.75       , 0         )$    & 8.00000 \\
2     &   $( 1 , 0.64864865 , -0.2972973 )$   & 9.25000 \\
3     &   $( 1 , 0.61756374 , -0.37110482)$   & 9.54054 \\
4     &   $( 1 , 0.60879835 , -0.38883968)$   & 9.59490 \\
5     &   $( 1 , 0.60641274 , -0.39309539)$   & 9.60407 \\
6     &   $( 1 , 0.60577683 , -0.39412075)$   & 9.60543 \\
7     &   $( 1 , 0.60560975 , -0.39436892)$   & 9.60557 \\
            \bottomrule
        \end{tabular}
    \end{table}

    故主特征值与其对应的特征向量为
    \begin{equation*}
        \lambda_1\approx 9.605,\quad \mathbf{x}_1=( 1 , 0.60560975 , -0.39436892)^\top
    \end{equation*}
    \end{solution}

    \item[3.] \begin{solution}
        令
        \begin{equation*}
            \mathbf{B}=\mathbf{A}-6\mathbf{E}=\begin{pmatrix}
                0 & 2 & 1 \\ 2 & -3 & 1 \\ 1 & 1 & -5
            \end{pmatrix}
        \end{equation*}
        取\begin{equation*}
            \mathbf{P}=\begin{pmatrix}
                0 & 1 & 0 \\ 0 & 0 & 1 \\ 1 & 0 & 0 
            \end{pmatrix}
        \end{equation*}以避免对角线上的0元素,
        进行 LU 分解,$\mathbf{P}\mathbf{B}=\mathbf{L}\mathbf{U}$,有
        \begin{equation*}
            \mathbf{L}=\begin{pmatrix}
                1 & 0 & 0 \\ \frac{1}{2} & 1 & 0 \\ 0 & \frac{4}{5} & 1
            \end{pmatrix},\quad \mathbf{U}=\begin{pmatrix}
                2 & -3 & 1 \\ 0 & \frac{5}{2} & -\frac{11}{2} \\ 0 & 0 & \frac{27}{5}
            \end{pmatrix}
        \end{equation*}
        根据反幂法迭代公式,
        \begin{equation*}
            \begin{cases}
                \mathbf{L}\mathbf{y}_k=\mathbf{P}\mathbf{u}_{k-1} \\
                \mathbf{U}\mathbf{v}_k=\mathbf{y}_k\\
                \mathbf{u}_k=\frac{\mathbf{v}_k}{\max(\mathbf{v}_k)}
            \end{cases}
        \end{equation*}
        根据 $\mathbf{U}_1\mathbf{v}_1=(1,1,1)^\top$,有
        \begin{equation*}
            \mathbf{v}_1 = (1.61851852, 0.80740741, 0.18518519)^\top,\quad \mathbf{u}_1 = (1,        0.49885584, 0.11441648)^\top
        \end{equation*}
        则进行如下的迭代过程:
        \begin{table}[h]
            \centering
            \begin{tabular}{ccc}
                \toprule
                $k$ & $\mathbf{u}_k^\top$ & $\max(\mathbf{v}_k)$ \\
                \midrule
1     &   $(1,        0.49885584, 0.11441648)$    & 1.6185 \\
2     &   $(1      ,  0.5349076   ,0.2761807)$     & 0.74294 \\
3     &   $(1      ,   0.51810545 ,0.23348783)$    &  0.78759 \\
4     &   $(1      ,   0.52470794 ,0.24451802)$    &  0.77284 \\
5     &   $(1      ,   0.52225069 ,0.24155724)$    &  0.77757 \\
6     &   $(1      ,   0.52312807 ,0.24237021)$    &  0.77602 \\
7     &   $(1      ,   0.52282154 ,0.24214025)$    &  0.77653 \\
8     &   $(1      ,   0.522927   ,0.24220711)$    &  0.77636 \\
9     &   $(1      ,   0.52289107 ,0.24218715)$    &  0.77642 \\
10    &   $(1      ,   0.52290323 ,0.24219325)$    &  0.77640 \\
11    &   $(1      ,   0.52289914 ,0.24219135)$    &  0.77640 \\
                \bottomrule
            \end{tabular}
        \end{table}

        最终可得 $\lambda \approx \frac{1}{0.77640}+6=7.28799$,特征向量为 $(1      ,   0.52289914 ,0.24219135)$。
    \end{solution}

    \item[9(1).] \begin{solution}
        \begin{equation*}
            \mathbf{A}=\begin{pmatrix}
                1 & 2 & 0 \\ 2 & -1 & 1 \\ 0 & 1 & 3
            \end{pmatrix}
        \end{equation*}

        根据 QR 方法迭代公式,
        \begin{align*}
            \mathbf{A}_{k}&=\mathbf{Q}_k\mathbf{Q}_k\\
            \mathbf{A}_{k+1}&=\mathbf{R}_k\mathbf{Q}_k
        \end{align*}

        有如下的迭代过程:
        \begin{table}[H]
            \centering
            \begin{tabular}{cc|cc}
                \toprule
                $k$ & $\mathbf{A}_k$ & $k$ & $\mathbf{A}_k$\\
                \midrule
                1 &      $\begin{pmatrix}
                    1.00000 & -2.19089 &  0.00000\\
                   -2.19089 & -0.66667 &  1.19257\\
                    0.00000 &  1.19257 &  2.66667
                  \end{pmatrix}$ & 11 &     $\begin{pmatrix}
                    3.36420 & -0.21413 &  0.00000\\
                   -0.21413 & -2.34899 &  0.25820\\
                    0.00000 &  0.25820 &  1.98479
                  \end{pmatrix}$ \\
                  2 &      $\begin{pmatrix}
                    1.27586 &  2.33263 &  0.00000\\
                    2.33263 & -0.56957 &  1.20507\\
                    0.00000 &  1.20507 &  2.29371
                  \end{pmatrix}$ & 12 &     $\begin{pmatrix}
                    3.36830 &  0.15067 &  0.00000\\
                    0.15067 & -2.35745 &  0.21783\\
                    0.00000 &  0.21783 &  1.98915
                  \end{pmatrix}$ \\
                  3 &      $\begin{pmatrix}
                    1.81951 & -2.29349 &  0.00000\\
                   -2.29349 & -0.86236 &  1.06119\\
                    0.00000 &  1.06119 &  2.04285
                  \end{pmatrix}$ & 13 &     $\begin{pmatrix}
                    3.37031 & -0.10599 &  0.00000\\
                   -0.10599 & -2.36259 &  0.18375\\
                    0.00000 &  0.18375 &  1.99228
                  \end{pmatrix}$ \\
                  4 &      $\begin{pmatrix}
                    2.40694 &  2.00767 & -0.00000\\
                    2.00767 & -1.34810 &  0.88307\\
                    0.00000 &  0.88307 &  1.94116
                  \end{pmatrix}$ & 14 &     $\begin{pmatrix}
                    3.37131 &  0.07456 &  0.00000\\
                    0.07456 & -2.36581 &  0.15498\\
                    0.00000 &  0.15498 &  1.99450
                  \end{pmatrix}$\\
                  5 &      $\begin{pmatrix}
                    2.84138 & -1.59080 &  0.00000\\
                   -1.59080 & -1.76248 &  0.73085\\
                    0.00000 &  0.73085 &  1.92110
                  \end{pmatrix}$ & 15 &     $\begin{pmatrix}
                    3.37180 & -0.05244 &  0.00000\\
                   -0.05244 & -2.36789 &  0.13070\\
                    0.00000 &  0.13070 &  1.99609
                  \end{pmatrix}$\\
                  6 &      $\begin{pmatrix}
                    3.09885 &  1.18594 &  0.00000\\
                    1.18594 & -2.02892 &  0.60933\\
                    0.00000 &  0.60933 &  1.93006
                  \end{pmatrix}$ & 16 &     $\begin{pmatrix}
                    3.37204 &  0.03689 &  0.00000\\
                    0.03689 & -2.36926 &  0.11021\\
                    0.00000 &  0.11021 &  1.99722
                  \end{pmatrix}$\\
                  7 &      $\begin{pmatrix}
                    3.23554 & -0.85692 &  0.00000\\
                   -0.85692 & -2.18082 &  0.51118\\
                    0.00000 &  0.51118 &  1.94528
                  \end{pmatrix}$ & 17 &     $\begin{pmatrix}
                    3.37216 & -0.02595 &  0.00000\\
                   -0.02595 & -2.37019 &  0.09293\\
                    0.00000 &  0.09293 &  1.99802
                  \end{pmatrix}$ \\
                  8 &      $\begin{pmatrix}
                    3.30467 &  0.61009 &  0.00000\\
                    0.61009 & -2.26397 &  0.43031\\
                    0.00000 &  0.43031 &  1.95930
                  \end{pmatrix}$ & 18 &     $\begin{pmatrix}
                    3.37222 &  0.01825 &  0.00000\\
                    0.01825 & -2.37082 &  0.07836\\
                    0.00000 &  0.07836 &  1.99860
                  \end{pmatrix}$ \\
                  9 &      $\begin{pmatrix}
                    3.33897 & -0.43143 &  0.00000\\
                   -0.43143 & -2.30938 &  0.36279\\
                    0.00000 &  0.36279 &  1.97041
                  \end{pmatrix}$ & 19 &     $\begin{pmatrix}
                    3.37225 & -0.01284 & -0.00000\\
                   -0.01284 & -2.37125 &  0.06607\\
                    0.00000 &  0.06607 &  1.99900
                  \end{pmatrix}$\\
                  10 &     $\begin{pmatrix}
                    3.35588 &  0.30415 &  0.00000\\
                    0.30415 & -2.33460 &  0.30604\\
                    0.00000 &  0.30604 &  1.97873
                  \end{pmatrix}$ & 20 &     $\begin{pmatrix}
                    3.37227 &  0.00903 &  0.00000\\
                    0.00903 & -2.37156 &  0.05570\\
                    0.00000 &  0.05570 &  1.99929
                  \end{pmatrix}$\\
                  \bottomrule
            \end{tabular}
        \end{table}

        此时,
        \begin{equation*}
            \tilde{\mathbf{Q}}_{20}=\begin{pmatrix}
                0.28304 & -0.51361 & -0.80999\\
                0.33341 &  0.84455 & -0.41901\\
                0.89929 & -0.15146 &  0.41029
              \end{pmatrix}
        \end{equation*}
        列向量即为对应的特征向量。
    \end{solution}
    \item[10.] \begin{solution}
        \begin{equation*}
            \mathbf{A}=\begin{pmatrix}
                1 & 1 & 1 \\ 2 & -1 & -1 \\ 2 & -4 & 5
            \end{pmatrix}
        \end{equation*}
        $\alpha_1=(1,2,2)^\top$,范数 $\lVert \alpha_1\rVert_2=3$,取 $\mathbf{y}_1=(-3,0,0)^\top$,有 $\mathbf{u}_1=\frac{\alpha_1-\mathbf{y}_1}{\lVert \alpha_1 - \mathbf{y}_1 \rVert}=\frac{1}{\sqrt{6}}(2,1,1)^\top$,则
        \begin{equation*}
            \mathbf{H}_1=\mathbf{E}-2\mathbf{u}\mathbf{u}^\top = \begin{pmatrix}
                -\frac13 & -\frac23 & -\frac23 \\ -\frac23 & \frac23 & -\frac13 \\ -\frac23 & -\frac13 & \frac23
            \end{pmatrix}
        \end{equation*}
        则
        \begin{equation*}
            \mathbf{A}_2 = \mathbf{H}_1\mathbf{A}_1= \begin{pmatrix}
                -3 & 3 & -3 \\ 0 & 0 & -3 \\ 0 & -3 & 3
            \end{pmatrix}
        \end{equation*}
        $\tilde{\alpha}_2=(0,-3)^\top$,范数 $\lVert \tilde{\alpha_2}\rVert_2=3$,取 $\tilde{\mathbf{y}}_2=(-3,0)^\top$,有 $\tilde{\mathbf{u}}_2=\frac{\tilde{\alpha_2}-\tilde{\mathbf{y}}_2}{\lVert \tilde{\alpha_2}-\tilde{\mathbf{y}}_2 \rVert}=\frac{1}{\sqrt{2}}(1,-1)^\top$,则
        \begin{equation*}
            \tilde{\mathbf{H}}_2=\mathbf{E}-2\tilde{\mathbf{u}}_2\tilde{\mathbf{u}}_2^\top=\begin{pmatrix}
                0 & 1 \\ 1 & 0
            \end{pmatrix},\quad \mathbf{H}_2=\begin{pmatrix}
                1 \\ & \tilde{\mathbf{H}}_2
            \end{pmatrix}=\begin{pmatrix}
                1\\ & & 1\\ & 1
            \end{pmatrix}
        \end{equation*}
        故 
        \begin{equation*}
            \mathbf{A}_3=\mathbf{H}_2\mathbf{A}_2=\begin{pmatrix}
                -3 & 3 & -3 \\ & -3 & 3 \\ & & -3
            \end{pmatrix}=\mathbf{R}
        \end{equation*}
        而
        \begin{equation*}
            \mathbf{Q}=\mathbf{H}_1\mathbf{H}_2=\begin{pmatrix}
                -\frac13 &  -\frac23 & -\frac23  \\ -\frac23 &  -\frac13 & \frac23  \\ -\frac23 & \frac23 & -\frac13 
            \end{pmatrix}
        \end{equation*}
    \end{solution}
\end{itemize}

\end{document}